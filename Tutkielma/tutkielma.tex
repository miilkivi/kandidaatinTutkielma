% --- Template for thesis / report with tktltiki2 class ---

\documentclass[finnish]{tktltiki2}

% tktltiki2 automatically loads babel, so you can simply
% give the language parameter (e.g. finnish, swedish, english, british) as
% a parameter for the class: \documentclass[finnish]{tktltiki2}.
% The information on title and abstract is generated automatically depending on
% the language, see below if you need to change any of these manually.
% 
% Class options:
% - grading                 -- Print labels for grading information on the front page.
% - disablelastpagecounter  -- Disables the automatic generation of page number information
%                              in the abstract. See also \numberofpagesinformation{} command below.
%
% The class also respects the following options of article class:
%   10pt, 11pt, 12pt, final, draft, oneside, twoside,
%   openright, openany, onecolumn, twocolumn, leqno, fleqn
%
% The default font size is 11pt. The paper size used is A4, other sizes are not supported.
%
% rubber: module pdftex

% --- General packages ---

\usepackage[utf8]{inputenc}
\usepackage{lmodern}
\usepackage{microtype}
\usepackage{amsfonts,amsmath,amssymb,amsthm,booktabs,color,enumitem,graphicx}
\usepackage[pdftex,hidelinks]{hyperref}

% Automatically set the PDF metadata fields
\makeatletter
\AtBeginDocument{\hypersetup{pdftitle = {\@title}, pdfauthor = {\@author}}}
\makeatother

% --- Language-related settings ---
%
% these should be modified according to your language

% babelbib for non-english bibliography using bibtex
\usepackage[fixlanguage]{babelbib}
\selectbiblanguage{finnish}

% add bibliography to the table of contents
\usepackage[nottoc,numbib]{tocbibind}
% tocbibind renames the bibliography, use the following to change it back
\settocbibname{Lähteet}

% --- Theorem environment definitions ---

\newtheorem{lau}{Lause}
\newtheorem{lem}[lau]{Lemma}
\newtheorem{kor}[lau]{Korollaari}

\theoremstyle{definition}
\newtheorem{maar}[lau]{Määritelmä}
\newtheorem{ong}{Ongelma}
\newtheorem{alg}[lau]{Algoritmi}
\newtheorem{esim}[lau]{Esimerkki}

\theoremstyle{remark}
\newtheorem*{huom}{Huomautus}


% --- tktltiki2 options ---
%
% The following commands define the information used to generate title and
% abstract pages. The following entries should be always specified:

\title{Johtaminen ja johtajuus ohjelmistotuotantoprojekteissa, johtajan persoonallisuus}
\author{Mika Kivi}
\date{\today}
\level{Kandidaatintutkielma}
\abstract{Ohjelmistojen tuottamisessa käytettävät metodologiat ovat muuttuneet. Perinteisten menetelmien rinalle on kehittynyt uudempia ketteriä menetelmiä. Tuotantoprosessin muutos on vaikuttanut johtamisen roolin muutokseen projekteissa. Johtamiseen vaikuttavat johtajan persoonallisuus ja erilaiset johtamistyylit. Tutkielman tarkoituksena on tutustua ohjelmistotuotantoprojektien johtamiseen ja löytää avaimet tehokkaaseen johtamiseen ohjelmistotuotantoprojekteissa. Tehokkaaseen johtamiseen liittyy johtajan persoonallisuuden tutkiminen ja määrittäminen. }

% The following can be used to specify keywords and classification of the paper:

\keywords{Ohjelmistotuotanprojektin johtaminen, Johtajuus, Johtajan persoonallisuus, Ketterät menetelmät}
\classification{} % classification according to ACM Computing Classification System (http://www.acm.org/about/class/)
                  % This is probably mostly relevant for computer scientists

% If the automatic page number counting is not working as desired in your case,
% uncomment the following to manually set the number of pages displayed in the abstract page:
%
% \numberofpagesinformation{16 sivua + 10 sivua liitteissä}
%
% If you are not a computer scientist, you will want to uncomment the following by hand and specify
% your department, faculty and subject by hand:
%
% \faculty{Matemaattis-luonnontieteellinen}
% \department{Tietojenkäsittelytieteen laitos}
% \subject{Tietojenkäsittelytiede}
%
% If you are not from the University of Helsinki, then you will most likely want to set these also:
%
% \university{Helsingin yliopisto}
% \universitylong{HELSINGIN YLIOPISTO --- HELSINGFORS UNIVERSITET --- UNIVERSITY OF HELSINKI} % displayed on the top of the abstract page
% \city{Helsinki}
%


\begin{document}

% --- Front matter ---

\maketitle        % title page
\makeabstract     % abstract page

\tableofcontents  % table of contents
\thispagestyle{empty}
\newpage          % clear page after the table of contents


% --- Main matter ---

\setcounter{page}{1}
\pagenumbering{arabic}
\section{Johdanto}


Ohjelmistotuotanto on ihmislähtöistä toimintaa, jossa ihmiset tekevät ohjelmistoja ihmisille~\cite{Wang:2010:PPP:1810295.1810302}. Ihmisten toiminnalla on suuri vaikutus projektien onnistumiseen ja epäonnistumiseen~\cite{Wang:2009:PMP:1639950.1640049}. Muita projektin tulokseen vaikuttavia tekijöitä ovat projektin toimintatavat, käytettävä kehitysmenetelmä ja ympäristö, jossa projekti toteutetaan~\cite{McLeod:2011:FAS:1978802.1978803}.

Ohjelmistotuotantoprojekteissa on tapahtumassa murros perinteisistä menetelmistä kohti uudempia ketteriä menetelmiä~\cite{Chow2008961}. Kehitysprosessin muuttuminen on vaikuttanut projektiryhmän toimintaan. Muutokset ovat vaikuttaneet ohjelmistotuotantoprojektien johtamiseen sekä johtajana toimimiseen ohjelmistotuotantoprojekteissa.

Perinteisessä vesiputousmallissa johtaminen on ollut kontrolloivaa ja johtamistyyli voidaan rinnastaa armeijassa käytettävään johtamiseen~\cite{Nerur:2005:CMA:1060710.1060712}. Johtajan ja alaisen suhde perustuu perinteisissä menetelmissä sopimukseen, jossa alaiselle on määritelty tehtävä, josta hän saa korvauksen tehtävän suoritettuaan~\cite{thite2000leadership}. Johtajan tehtävänä on ollut kontrolloida, suunnitella ja tarkastella tuotantoprosessin toimivuutta~\cite{palmer2001emotional}.

Johtaminen on muuttunut viimeisten vuosikymmenien aikana. Teollisuuden muuttuminen palvelu orientointuneemmaksi on vaikuttanut johtajan tehtäviin. Johtajan tehtäviksi ovat muodostuneet työntekijöiden motivaation kasvattaminen ja inspiraation lisääminen työryhmän keskuudessa~\cite{palmer2001emotional}. Ohjelmistotuotantoprojekteissa tähän muutokseen on vaikuttanut uudempien menetelmien yleistyminen. Ketterien menetelmien johtamisessa korostetaan yhteistyötä ja johtajuutta~\cite{Nerur:2005:CMA:1060710.1060712}.
 
Tässä tutkielmassa tarkastellaan ohjelmistotuotantoprojektien johtamista sekä johtajan persoonan ja johtamistyylin vaikutusta ohjelmistotuotantoprojekteihin ja niiden onnistumiseen. Johtajuuteen ja johtamiseen vaikuttavat johtotehtävissä olevien henkilöiden persoonallisuudet. Ohjelmistotuotantoprojektien kehitysmenetelmien muutos on vaikuttanut johtamistyylien vaikutukseen ja kehittymiseen ohjelmistotuotannossa. 

Tutkielma koostuu kolmesta kä\-sit\-te\-ly\-lu\-vus\-ta. Ensimmäisessä kä\-sit\-te\-ly\-lu\-vus\-sa käsitellään ketterien menetelmien ja johtajuuden käsitteistöä. Luvussa tarkastellaan myös kuinka kehitysmenetelmien muutos on vaikuttanut ohjelmistotuotantoprojektien johtamiseen. Toisessa käsittelyluvussa paneudutaan ohjelmistotuotantoprojektien johtamiseen vaikuttaviin tekijöihin, kuten projektiryhmän johtaminen, ylemmän hallinnon vaikutus, ryhmän kokoaminen ja muutoksen toteuttaminen. Näiden kahden luvun pohjalta viimeisessä  käsittelyluvussa tarkastellaan johtajan persoonallisuutta ja johtamistavan vaikutusta ohjelmistotuotantoprojekteissa. Käsittelyssä ovat menetelmät, joiden avulla projektin johtajalla on mahdollisuus tunnistaa johtamiseen vaikuttavia persoonallisuuden piirteitä.


\section{Ketterät menetelmät ja johtaminen}

Tässä luvussa tutustutaan tutkielmassa käsiteltäviin aihepiireihin. En\-sim\-mäi\-sek\-si käsitellään ketteriä menetelmiä ja niiden perusideoita. Samalla luodaan katsaus scrumin ja XP:n käytäntöihin sekä johtamisen ilmenemiseen näissä menetelmissä.  Luvun jälkimmäinen osa selventää, mitä johtaminen on tämän tutkielman näkökulmasta.  


\subsection{Ketterien menetelmien perusteita}

Ketterät menetelmät (agile methodologies) ovat kokoelma erilaisia kehitysmenetelmiä, joiden ideologia perustuu "Ketterän manifestin" ajatuksiin~\cite{fowler2001agile}. Menetelmien lähtökohtana on pyrkiä mahdollisimman korkeaan asiakastyytyväisyyteen. Tavoitteen saavuttamiseksi ohjelmistosta pyritään tarjoamaan asiakkaalle toimiva versio mahdollisimman nopeasti ja mahdollisimman usein kehityksen aikana~\cite{fowler2001agile}.

Manifesti on syntynyt vuonna 2001 17 tunnetun ohjelmistoalan ammattilaisen tapaamisessa. Ryhmän jäsenten tarkoitus oli löytää yhteinen käsitys uusista menetelmistä, joita ohjelmistojen kehitykseen oli syntynyt. Ketteriksi menetelmiksi voidaan lukea muun muassa Extreme programing (XP), scrum, Crystal ja Feature-Driven development~\cite{fowler2001agile}.  Näistä keskitymme tarkemmin johtamiseen XP- ja scrum-menetelmissä.

Ketterät menetelmät perustuvat iteratiiviseen kehitykseen, jossa kehitystä tehdään iteraatiossa~\cite{cohen2004introduction}. Iteraation jälkeen tarkastellaan tuotosta ja päätetään, mitä seuraavassa iteraatiossa toteutetaan. Samalla myös tarkastellaan kehitysprosessin toimivuutta ja prosessia voidaan kehittää parempaan suuntaan~\cite{cohen2004introduction}.


Ketterät menetelmät, kuten XP ja scrum perustavat työskentelyn ryh\-mä\-työl\-le~\cite{4755768}, jossa oleellisessa roolissa on ihmisten välinen yhteistyö. Ryh\-mä\-työs\-ken\-te\-ly koetaan tehokkaaksi tavaksi toteuttaa monimutkaisia projekteja, joihin Ohjelmistotuotantoprojektit voidaan lukea, koska kesken projektin tapahtuvat muutokset ovat yleisiä.

Scrum-menetelmän periaatteet esiteltiin ensimmäisen kerran vuonna 1986 artikkelissa "The New New Product Development Game"~\cite{nonaka1986new}. Scrumissa johtaminen jakautuu virallisesti kahdelle henkilölle. Ryhmää johtavilla henkilöillä on erilaiset nimitykset tehtävänsä mukaan, scrummaster ja tuoteomistaja (product owner)~\cite{4755768}. Scrummasterin rooliin liittyviä tehtäviä ovat ryhmän kokoaminen, osallistuminen päätöksentekoon scrumtiimin osana ja työrauhan luominen scrumtiimille~\cite{4755768}. Joissain organisaatioissa ryhmän toimintaan vaikuttaa myös kolmas johtaja, jota ei ole määritelty scrumin periaatteissa~\cite{4755768}. Tutkimuksessa kolmannesta johtajasta on käytetty nimitystä projektimanageri. Johtajien rooleihin paneudutaan tarkemmin myöhemmin tässä tutkielmassa. 

Scrumtiimin kooksi on määritelty 1-7 henkilöä ja iteraation, josta scrumissa käytetään nimitystä sprintti, pituudeksi 40 päivää~\cite{cohen2004introduction}. Menetelmä on kuitenkin kehittynyt, esimekiksi Schwaber ilmoittaa yhden scrumtiimin kooksi 3-6 henkilöä ja sprintin pituudeksi 1-4 viikkoa~\cite{schwaber1995scrum}. Scrumtiimi on näiden lähteiden perusteella suhteellisen pieni koostuessaan alle kymmenestä henkilöstä. Scrummasterin ja tuoteomistajan lisäksi ryhmästä löytyy useita rooleja: testaaja, kehittäjä ja dokumentoija~\cite{schwaber1995scrum}. Periaatteiden mukaan jokainen ryhmän jäsen osallistuu kaikkiin edellä mainituista rooleista.

Extreme Programing (XP) on kehittänyt Kent Beck, joka artikkelissaan "Embracing change with extreme programming"  \ esittelee menetelmän ideologiaa ja periaatteita~\cite{796139}. XP:ssä ohjeet toiminnalle ovat tarkemmat kuin scrumissa, mutta menetelmän periaatteen mukaan ne ovat vain sääntöjä~\cite{cohen2004introduction}. Ryhmällä on oikeus muuttaa sääntöjä kehittääkseen prosessia~\cite{cohen2004introduction}. Tästä muutoksesta ja muutoksen toteuttamisesta puhutaan myöhemmin tässä tutkielmassa.

Projektiryhmän työskentelytila on XP:ssä avoin, missä keskellä tilaa ovat tietokoneet, joilla ohjelmointi toteutetaan. Ohjelmointi toteutetaan pariohjelmointina. Jokaisella parilla on vain yksi näyttö, näppäimistö ja hiiri. Ohjelman osien kehitystä ei ole osoitettu tietyille henkilöille vaan jokaisella ryhmän jäsenellä on oikeus muuttaa ohjelman kaikkia osa-alueita. Testauksen merkitystä projektin onnistumisessa korostetaan XP:ssä. Asiakas osallistuu projektiin täyspäiväisesti eli on osa kehitysryhmää~\cite{796139}. XP:ssä ryhmän koko on 2-10 henkilöä ja pyrähdyksen pituus on 2 viikkoa~\cite{cohen2004introduction}. 

XP:ssa johtajan roolia ei ole määritelty määritelmän puitteissa. XP on enemmänkin kehitysryhmän sisäinen yksinkertainen säännöstö, kuinka toimitaan tai kuinka tulisi toimia, jotta kehitys onnistuisi tehokkaasti~\cite{Augustine:2005:APM:1101779.1101781}. XP voidaankin pitää ohjelmiston tuotanto metodologiana, eikä niinkään ohjelmiston tuottamisen hallintametodologiana~\cite{cohen2004introduction}. Ryhmänjohtajan kannalta on tärkeää seurata, kuinka XP:n periaatteet toteutuvat ryhmän toiminnassa. Johtajan yhtenä tehtävänä  ketterissä menetelmissä voidaan pitää mahdollisten esteiden poistamista, mitkä estävät menetelmän periaatteiden toimimisen projektissa~\cite{Augustine:2005:APM:1101779.1101781}.

\subsection{Johtajuus ja hallinta}

Johtaminen on käsitteenä erittäin laaja ja johtaminen on tilannesidonnaista. Tutkimuksissa johtamisesta puhutaan monin eri käsittein ja jokaisen suuntauksen käsitteleminen ei ole tämän tutkielman osalta oleellista. Tässä osassa tutustumme johtamisen käsitteistöön, jonka avulla johtamisen tarkasteleminen on helpompaa tutkielman loppuosassa.

Johtamista esiintyy ryhmätilanteissa ja yleisemmin tilanteissa, joissa ihmiset vaikuttavat toistensa toimintaan. Tässä tutkielmassa johtamista käsitellään kahdesta näkökulmasta. Hallinta (management) tarkoittaa johtamiseen liittyviä käsitteitä kuten suunnittelu, organisointi, henkilöstöhallinto ja organisaation kontrollointi. Johtajuus (leadership) tarkoittaa prosessia, jonka avulla henkilö voi tukea ja auttaa muita, jotta yhteinen tavoite saavutetaan.

Johtaessaan ryhmäänsä johtajalle syntyy seuraajia. Johtaja ei voi pakottaa alaisiaan seuraajikseen vaan johtajan tulee luoda visio, jonka takia seuraajat tahtovat seurata johtajaansa~\cite{raccoon2006leadership}. Kaikista johtajan alaisista ei välttämättä tule johtajansa seuraajia vaan johtajan ja alaisen suhde voi säilyä hallinnallisella tasolla~\cite{raccoon2006leadership}. Projektista vastaava henkilö ei ole johtaja, jollei hänellä ole seuraajia.

Hallinnan ja johtajuuden ero on niiden suunta. Hallinnassa johtajan ja alaisen välinen suhde on ylhäältä alaspäin suuntautunutta~\cite{raccoon2006leadership}. Johtajuudessa seuraajat oikeuttavat johtajan johtajuuden, joten johtaminen on suuntautunut alhaalta ylöspäin~\cite{raccoon2006leadership}. Johtajuudessa seuraajat tarvitsevat johtajaansa, kun hallinnossa ylemmän tason johtajat tarvitsevat alemman tason johtajiaan~\cite{raccoon2006leadership}.

Johtajuudessa voidaan erottaa kaksi erilaista suuntausta. Johtajan auktoriteetti ja johtaminen voivat perustua palkkiohin ja tuloksiin johtajan ja työntekijän välillä~\cite{bass1990transactional}. Tätä johtamista kutsutaan tutkielmassa transaktiojohtamiseksi (transactional leadership). Toisessa johtamistavassa johtaja pyrkii innostamaan ja luomaan avoimuutta työntekijöiden keskuudessa, jotta työntekijät toimivat ryhmän hyväksi eivätkä omaksi edukseen~\cite{bass1990transactional}. Johtamistavasta käytetään nimitystä  transformaatiojohtaminen (transformational leadership).

Transaktiojohtamisessa suhde työntekijän ja johtajan välillä perustuu sopimukseen~\cite{bass1990transactional}. Sopimuksessa määritellään työntekijän tehtävät, jotta hän ansaitsee hänelle tarkoitetun palkkion. Työntekijän toiminta perustuu luvattuihin palkkioihin, joita saa hyvin suoritetusta työstä, tai sanktiohin, joita työntekijä saa poiketessaan annetuista ohjeista~\cite{bass1990transactional}. Transaktiojohtaminen voi onnistua, jos johtajalla on mahdollisuus tarjota palkkioita, joilla on merkitystä alaisilleen~\cite{bass1990transactional}.

Transaktiojohtamisessa esimiehen ja alaisen suhde ei perustu johtaja-seuraaja -suhteeseen. Transaktiojohtaminen on yleisempää hallinnon tasolla, mutta myös hallinnossa johtamisesta tulee tehokkaampaa, kun esimiehellä ja alaisella on johtaja-seuraaja -suhde. Seuraajilla on sama visio yrityksen toiminnasta ja suunnasta. Johtaja-seuraaja -suhde edistää koko organisaation kulkemista samaan suuntaan.

Transformaatiojohtamisessa johtajan täytyy saavuttaa alaisensa luottamus persoonallisuutensa ja toimintansa avulla. Luottamuksen syntymiseen vaikuttavat johtajan karisma, kyky huomioida alaisten emotionaaliset tarpeet ja kyky älyllisesti stimuloida alaisiaan~\cite{bass1990transactional}. Karismaattinen johtaja inspiroi alaisiaan siten, että alaiset saavuttavat suuria saavutuksia pienellä panostuksella työhönsä~\cite{bass1990transactional}.

Transformaatiojohtaja kykenee huomioimaan alaistensa emotionaaliset tarpeet. Johtajalla on kyky käsitellä alaistensa erilaiset tarpeet ja samalla mahdollistaa alaisilleen parempi tunne työn tekemisestä~\cite{palmer2001emotional}. Huomioimalla alaisensa ihmisinä transformaatiojohtaja saavuttaa todennäköisesti johtaja-seuraaja -suhteen, joka tehostaa toimintaa ryhmän tavoitteen hyväksi~\cite{raccoon2006leadership}.

Transformaatiojohtaminen koetaan tehokkaammaksi kuin transaktiojohtaminen~\cite{palmer2001emotional}. Tämän suuntainen vaikutus on havaittu useissa eri yhteyksissä, joissa johtamista esiintyy. Esimerkkinä voidaan mainita uskonnolliset yhteisöt~\cite{bass1990transactional}. Kun johtaja on transformaatiojohtaja, hänen alaisensa eivät vain työskentele tehokkaammin vaan alaiset ovat myös tyytyväisiä organisaation toimintaan. Transformaatiojohtamisen olettaisi sopivan myös ohjelmistotuotantoprojektin johtamiseen ketteriä menetelmiä käytettäessä.

IT-projekteissa molempiin johtamistapoihin liittyy käsite tekninen johtaminen. Teknisellä johtamisella tarkoitetaan johtajan teknisiä kykyjä. Tekniset kyvyt viittaavat johtajan taitoihin kuten insinööritaito, informaatioteknologiataidot ja tutkimus- ja kehitystaidot~\cite{thite2000leadership}. Teknisesti suuntautuneet henkilöt ovat usein enemmän kiinnostuneita ammattitaitonsa kehittämisestä kuin organisaation menestyksestä. 

   



  


\section{Johtaminen ohjelmistotuotantoprojekteissa}

Ohjelmistotuotantoprojektiin vaikuttavat johtajat organisaation eri tasoilta. Ketterässä kehityksessä kehitysryhmät ovat  itsestään organisoituvia ja ryhmän johtajan tulee luottaa kehitysryhmän tietoihin ja taitoihin~\cite{fowler2001agile}.

Luvussa tutustutaan tutkimukseen, jota on suoritettu ohjelmistotuotantoprojektien johtamiseen liittyen. Käsittelyssä ovat kehitysryhmän johtaminen sekä ylemmän organisaation vaikutus johtamiseen ja ryhmän toimintaan. Luvun lopussa käsitellään ohjelmistotuotantoprojektin kannalta kahta tärkeää osa-aluetta: kehitysryhmän kokoamista ja riskien hallintaa. Riskien hallintaa tutkitaan kahdelta kannalta. Kuinka reagoida riskeihin ja mitä työkaluja riskeihin reagointiin voidaan käyttää. Toinen puoli riskien hallinnasta on muutosjohtaminen, joka sisältää koko kehitysprosessiin liittyvän muutoksen johtamisen.

\subsection{Kehitysryhmän johtaminen}

Ohjelmistotuotantoprojekteissa johtajan tehtävänä on muiden motivointi. Johtajan tehtäviin voidaan lukea esimerkiksi organisointi, innovointi, arviointi ja projektin onnistumisesta vastaaminen~\cite{4017705}, sekä resurssien hallinta ja projektin aikataulutus ~\cite{Dhomne:2012:ITL:2382887.2382899}. Projektin johtajan yhtenä tehtävänä voidaan pitää välikätenä toimimista ryhmän ja ulkoisten osallistujien välillä~\cite{McLeod:2011:FAS:1978802.1978803}. Johtajan ammattitaidolla ja projektin johtamisessa käytettävillä työkaluilla on suuri vaikutus ohjelmistotuotantoprojektin tuloksellisuuteen~\cite{McLeod:2011:FAS:1978802.1978803}. Tehoton johtaja sabotoi kehitysryhmän tuottavuutta~\cite{bradley1997effect}.

Hyvästä johtamisesta huolimatta kehitysryhmä saattaa jäädä tavoitteestaan, jolloin projektin onnistumisesta vastaava johtaja ei saavuta hänelle asetettua tavoitetta. Vaikka projekti epäonnistuisi tulee johtajan toiminan seuraajiensa kanssa olla rakentavaa ja positiivista~\cite{raccoon2006leadership}. Johtajan ei kannata vierittää kaikkea vastuuta epäonnistumisesta seuraajilleen. Johtajan tulee pyrkiä löytämään rakentavat ratkaisut tapahtuneisiin virheisiin yhdessä ryhmänsä kanssa.

Ketterissä menetelmissä johtajan valtaa on siirretty johtajalta ryhmälle ja johtaminen on muuttunut yhteistyön suuntaan armeijamaisesta kontrollointi johtamisesta~\cite{Nerur:2005:CMA:1060710.1060712}. Scrummasterin, jolla scrumissa tarkoitetaan projektiryhmän johtajaa, tehtävänä on suunnitella ryhmän kokoonpano ja ryhmälle säännöstö~\cite{4755768}. Scrummasterin tulee johtaa ryhmän palavereita niin, että jokainen ryhmän jäsen pääsee osallistumaan ryhmän toimintaan, jotta yhteisymmärrys tilanteissa saavutetaan~\cite{bradley1997effect}. Scrummasterilla ei ole ryhmäänsä välttämättä suoraa käskyvaltaa vaan hän pyrkii vaikuttamaan toimintaan seurattavan prosessin kautta.

Scrumissa johtamisen vastuuta on jaettu tuotteenomistajalle, jonka tehtävänä on vastata projektin tuotoksen hallinnasta, tuotteen kehitysjonosta (product backlog) ja osallistua projektin kontrolloinntiin~\cite{4755768}. Tuotteen kehitysjono on lista toteutettavista ominaisuuksista, josta vastaa tuoteomistaja. Tuotteen kehitysjonon avulla tuoteomistaja voi vaikuttaa projektin etenemiseen. Jonosta kehitysryhmä valitsee toteutettavat asiat ennen sprintin alkua. Ryhmä valitsee tuoteomistajan tärkeimmäksi arvioimat vaatimukset, jotka ryhmä toteuttaa sprintin aikana. Muuttamalla prioriteettijärjestystä tuotteen kehitysjonossa pystyy tuoteomistaja vaikuttamaan projektin tuotekehitykseen. Scrummasterin tehtäviin kuuluu avustaa tuoteomistajaa tuotteen kehitysjonon ylläpidossa~\cite{Nerur:2005:CMA:1060710.1060712}. 

Tutkimuksen mukaan teknisesti taitava tuoteomistaja saattaa olla haitaksi kehitysryhmän toiminnalle~\cite{Nerur:2005:CMA:1060710.1060712}. Liian tarkaksi muotoillut toiminnallisuuksien kuvaukset tuotteen kehitysjonossa voivat johtaa ryhmän tehokkuuden heikkenemiseen~\cite{Nerur:2005:CMA:1060710.1060712}. Tutkimuksessa on tärkeää huomata kulttuurin vaikutus ryhmän toimintaa, joten kyseistä tulosta ei voida yleistää eri kulttuureissa toimiviin projekteihin.

Ketterät menetelmät korostavat kehitysryhmän merkitystä projekteissa. Scrumissa ryhmän sisäiseen johtamiseen liittyy oleellisesti kaksi käsitettä itseorganisoituvat ryhmät ja jaettu johtaminen~\cite{4755768}. Itseorganisoitumisella tarkoitetaan ryhmän sisäistä tehtäväjakoa. Ulkopuolisten tekijöiden vaikutus ryhmän toimintaan ja päätöksentekoon sprintin aikana minimoidaan. Itseorganisoituminen ei tarkoita tilannetta, jossa projektiryhmää ei kontrolloida lainkaan, vaan ylemmän hallinnon tulee asettaa ryhmälle välietappeja, joiden avulla ryhmän edistymistä voidaan seurata~\cite{Nerur:2005:CMA:1060710.1060712}. Scrumissa välietappeja ovat sprinttien lopuksi pidettävät sprinttikatselmukset, joissa tuoteomistaja saa käsityksen, mitä sprintin aikana on toteutettu~\cite{schwaber1995scrum}.

Scrumryhmissä johtamisen tulisi olla jaettua. Päätöksentekotilanteissa johtajana tulisi toimia henkilön, jolla on käsiteltävästä aihealueesta paras asiantuntemus~\cite{4755768}. Jaettu johtaminen vaatii ryhmän jäseniltä kykyä huomioida erilaisten ihmisten mielipiteitä ja vaikutusta päätöksiin. Jotta jaettu johtaminen olisi tehokasta, tulee ryhmän toiminnassa korostua oppimisen ilmapiiri~\cite{4755768}. Jatkuvan kouluttamisen kautta projektin johtajat pystyvät kehittämään projektiin osallistuvien kehittäjien tietoja ja taitoja~\cite{dall2004project}. Kouluttamalla voidaan kehittää ryhmäläisten ryhmätyöskentelytaitoja.

Tehokas ja toimiva ryhmä on olennainen osa onnistunutta projektia. Projektin johtajalla on mahdollisuus vaikuttaa ryhmän muodostamiseen ja mahdollisuus parantaa ryhmän koheesiota. Parempi ryhmäkoheesio vaikuttaa positiivisesti ryhmän tuloksellisuuteen~\cite{bahli2005group, McLeod:2011:FAS:1978802.1978803}. Ryhmän toimintaan positiivisesti vaikuttavia tekijöitä ovat yhteinen tavoite, avoin kommunikaatio, nopeat ja rakentavat konfliktien ratkaisut, yleinen luottamus, uskominen synergiaan, avustaminen, kunnioitus ja hyvä johtaja~\cite{4017705}. Vaikka ryhmällä olisi hyvä koheesio voi ryhmä ajautua konflikteihin, mutta hyvä koheesio auttaa nopeaan konfliktien ratkaisuun~\cite{bradley1997effect}. Ristiriitatilanteista selvitään ilman kenenkään loukkaantumista kun ryhmä tukee toisiaan erilaisissa ongelmatilanteissa~\cite{bradley1997effect}.







\subsection{Ylemmän hallinnon vaikutus}

Ohjelmistotuotanprojektiryhmä on osa suurempaa organisaatiota, jonka alaisena projekti toimii. Organisaation ylemmällä hallinnolla on vaikutusta ohjelmistotuotantoprojektin johtamiseen ja tuloksellisuuteen. Päätöksiä tehdessään ylemmän hallinon tulee huomioda useita tekijöitä, joiden avulla hallinto voi vaikuttaa projektin tuloksellisuuteen.  Organisaation ylemmällä hallinnolla tarkoitetaan organisaation hallitusta ja toimitusjohtajaa, joiden vastuulla on organisaation strateginen suunta~\cite{McLeod:2011:FAS:1978802.1978803}.

Ylempi hallinto vaikuttaa ohjelmistotuotantoprojektin budjettiin ja aikatauluun~\cite{McLeod:2011:FAS:1978802.1978803}. Kiireellinen aikataulu ja tiukka budjetti ovat syitä ohjelmistotuotantoprojektin epäonnistumiselle. Seamanin ja Goun tutkimuksessa ~\cite{Guo:2008:SSP:1414004.1414046} projektin johtajat ilmoittivat aikataulun ja budjetin kolmen tärkeimmän syyn joukkoon prosessin muutokselle kesken projektin.

Organisaation ylempi hallinto tarjoaa projektin johtajalle käyttöön tietyt henkilöstöresurssit. Projektin johtaja joutuu valitsemaan kehitysryhmän ylemmän organisaation tarjoamista kehittäjistä. Projektin kehitysryhmästä tulee harvoin optimaalinen henkilöstön osalta~\cite{Dhomne:2012:ITL:2382887.2382899}. Ylempi hallinto vaikuttaa myös asiakkaan osallistumiseen projektin toteutukseen~\cite{McLeod:2011:FAS:1978802.1978803}. XP:tä käyttävissä organisaatiossa on XP:n periaatteiden mukaan määritetty, että asiakas on osa kehitysryhmää~\cite{796139}.

Organisaatiolla on usein vakiintuneet kehitysmenetelmät, joita organisaatiossa käytetään~\cite{McLeod:2011:FAS:1978802.1978803}. Asennoituminen työntekijöihin ja työntekijöiden asennoitumiseen työhön vaikuttaa organisaation yleinen ilmapiiri. Samalla heikko johtaminen ylemmältä tasolta vaikuttaa kehitysryhmään ja kehitysryhmän johtamiseen. Vakiintuneilla kehitysmenetelmillä on vaikutusta projektiryhmän johtajan toimintaan koska muutosjohtamistilanteissa kehitysmenetelmä sitoo ryhmän johtajan toimintamahdollisuuksia.

Ryhmän toimintaa kontrolloimalla ylemmän hallinnon edustaja ei saavuta haluttua tulosta. Ylemmän hallinnon edustajat usein unohtavat tehtävänsä luoda järjestystä kun he lisäävät kontrollia ryhmän toimintaan~\cite{Augustine:2005:APM:1101779.1101781}. Ammattitaitoiset henkilöt eivät usein sopeudu hallinnon työkaluihin ja tekniikoihin ellei niitä käytetä oikein ja tehokkaasti~\cite{Augustine:2005:APM:1101779.1101781}.

Organisaation ylempi johto voi asettaa rangaistusmenetelmiä, joiden avulla voidaan hallita tuotettavien ohjelmistojen laatua. Yleinen käsitys rangaistuksista on että ne laskevat kehittäjien motivaatiota. Wang ja Zhang kirjoittavat tutkimuksessaan, jossa on tutkittu kiinalaisen ohjelmistoyrityksen toimintaa ~\cite{Wang:2010:PPP:1810295.1810302}, että rangaistus politiikka voi tehostaa ohjelmistotuotantoprojektin toimintaa ja tuotteen laatua. Wang ja Zhang antavat johtajalle kolme ohjetta joiden avulla rangaistuksista saadaan toimivia. Rangaistus politiikan tulee olla oikeudenmukaista ja hyvin suunniteltua. Suunnittelussa hyvä ohjeistus on tärkeää. Kehittäjät tulee saada uskomaan että menetelmän tarkoituksena ei ole säästää rahaa pienentämällä palkkoja, vaan saada kehittäjistä motivoituneempia. Kolmas ja tutkijoiden mielestä hyvin oleellinen sääntö on, että rangaistus ei saa ylittää tiettyä osuutta kehittäjien palkasta. Esimerkiksi maksimirangaistus viisi prosenttia kuukausipalkasta.



\subsection{Ryhmän kokoaminen}


Ohjelmistotuotantoprojektin johtamisen onnistumisessa tärkeässä roolissa on projektin alussa tapahtuva ryhmän kokoaminen. Ohjelmistotuotantoprojekteissa johtajan kokoamiseen käyttämillä perusteilla on vaikutusta projektien onnistumiseen~\cite{daSilva2012}. Kootessaan ryhmäänsä johtajan tulee huomioda useita seikkoja, jotka vaikuttavat projektiryhmän toimintaan.  

Projektin johtajan tulee huomioida projektiryhmän koko muodostaessaan ryhmää. Projektiryhmän koko voi määräytyä ylemmän organisaation puolelta~\cite{McLeod:2011:FAS:1978802.1978803}. Projektiryhmä joka koostuu suuresta kehittäjäjoukosta koetaan riskitekijäksi projektin kannalta~\cite{McLeod:2011:FAS:1978802.1978803}. Ketterissä menetelmissä jos ryhmän koko on suuri esimerkiksi 15 henkilöä voi projektijohtaja jakaa ryhmän pienempiin osaryhmiin, jotta kehittämisestä tulee tehokkaampaa~\cite{Augustine:2005:APM:1101779.1101781}. Projektiryhmän koon lisäksi johtajan tulee huomioida ryhmäläisten kokemus aiemmasta keskinäisestä toiminnasta. Ihmiset, jotka eivät ole toimineet ennen yhdessä koetaan ryhmän kannalta riskitekijöiksi~\cite{McLeod:2011:FAS:1978802.1978803}. 

Da Silva ja kumppanit tutkivat projektiryhmän kokoajien syitä henkilöiden valinnalle ja niiden vaikutusta ohjelmistotuotantoprojektin onnistumiseen~\cite{daSilva2012, francca2009quantitative}. Ohjelmistotuotantoprojekteissa ryhmän kokoamisesta vastaa usein projektipäällikkö tai henkilöstöhallinnosta vastaava henkilö. Joissakin organisaatiossa ryhmän kokoamisen vastuu voi olla myös scrummasterilla. Projektiryhmän kokoajasta käytetään tässä tutkielmassa nimitystä projektipäällikkö.

Tutkimuksensa ensimmäisessä vaiheessa da Silva ja kumppanit tutkivat kahdeksaa kriteeriä, joiden perusteella projektipäälliköt valitsevat henkilöitä projekteihin~\cite{francca2009quantitative}. Myöhemmässä tutkimuksessa he kasvattivat kriteereiden määrän kymmeneen~\cite{daSilva2012}. Kriteerit ovat persoona, käyttäytyminen, tekninen profiili, asiakkaan tärkeys, tuotoksellisuus, käytettävyys, yksilön hinta projektille, projektin tärkeys, suositukset ja tehtävään sopivuus. Tarkemmat kuvaukset kriteereistä voi lukea da Silvan ja kumppaneiden tutkimuksesta~\cite{daSilva2012}.

Tutkimuksessaan da Silva ja kumppanit huomasivat että johtajat painottivat valinta kriteereissään teknistä profiilia ja projektin tärkeyttä. Tutkimuksessa kuitenkin huomattiin että suurin vaikutus projektin onnistumiseen oli persoonallisuudella ja käyttäytymisellä~\cite{daSilva2012}. Projektipäälliköt käyttävät usein entuudestaan tuttuja tapoja henkilöiden valitsemiseen. Osaamisen ja tiedon puute rekrytoinnissa vaikuttavat ihmistekijöiden käyttämiseen valintaprosessissa. Ohjelmistoyritysten tulisi huomioida ihmistekijöiden vaikutus ja kouluttaa johtajiaan huomioimaan henkilöresurssit~\cite{daSilva2012}. Ihmistekijöillä tarkoitetaan ihmisen persoonallisuutta ja toimintaa, mitkä vaikuttavat ihmisten toimintaan erilaisissa tilanteissa.

Rakentaessaan tehokasta ryhmää tulee projektipäällikön huomioida tär\-kei\-ksi todettuja tehokkaaseen ryhmään vaikuttavia tekijöitä. Bradley ja Hebert listaavat tekijöiksi artikkelissaan tehokkaan johtajuuden, ryhmän sisäisen kommunikaation, ryhmä koheesion ja ryhmäläisten persoonallisuuden heterogeenisyyden~\cite{bradley1997effect}. Ryhmän kokoamisvaiheessa johtaja voi huomioida ryhmäläisten persoonallisuuden ja pyrkiä löytämään ryhmäänsä erilaisia persoonallisuuksia. Kun ihmisillä on erilaisia tietoja ja taitoja on todennäköisempää että monimutkaiset ohjelmistotuotantoprojektit onnistuvat tavoitteissaan~\cite{bradley1997effect}.

Löytääkseen oikeat henkilöt oikeaan paikkaan ohjelmistotuotantoprojektien johtaja voi käyttää apunaan "I Opt" menetelmää ~\cite{Dhomne:2012:ITL:2382887.2382899, kliem1996teambuilding}. Menetelmässä ryhmän jäsenet arvioidaan kyselyn perusteella. Jokainen ryhmäläinen asetetaan yhteen neljästä luokkasta persoonallisuutensa ja ominaisuuksiensa mukaan. Nämä neljä luokkaa ovat toimintaan suuntautunut, looginen prosessoija, analysoija ja innovoija~\cite{ kliem1996teambuilding}. Johtajan ja ryhmän valinta olisi optimaalista jos jokaiseen tehtävään voitaisiin valita tehtävätyypiltään sopivimmat henkilöt~\cite{Dhomne:2012:ITL:2382887.2382899}.

Ryhmää kootessaan projektinjohtajalle on höytyä tunnistaa erilaiset persoonallisuustyypit, jotka hän valitsee projektiin." I Opt" menetelmän rinnalla toinen menetelmä henkilöiden persoonallisuuksien arviontiin on MBTI (Myers Briggs Type Indicator)~\cite{bradley1997effect}. MBTI menetelmästä ja sen käytöstä voi lukea tarkemmin artikkelista~\cite{myers1985manual}. MBTI:n ideologiaan paneudutaan myös tarkemmin myöhemmin tässä tutkielmassa kun puhutaan johtajan persoonallisuudesta ja sen vaikutuksesta ohjelmistotuotantoprojekteihin.

Menetelmässä henkilöä arvioidaan neljän tyypin perusteella. Henkilö on joko ekstrovertti tai introvertti. Henkilö perustaa päätöksensä joko järkeen tai intuitioon. Henkilö päättää, joko ajatuksillaan tai tunteillaan. Henkilö haluaa elää järjestetyssä ympäristössä tai kaaoksen rajamailla. Näistä ominaisuuksista saadaan muodostettua 16 erilaista profiilia henkilöille. Johtajan on hyvä huomioida se että jokainen näistä kombinaatiosta voi olla hyödyllinen projektille riippuen projektiryhmän kokonaisuudesta~\cite{bradley1997effect}. Bradleyn ja Herbertin tutkimus, jossa he käyttivät MBTI:tä selvittääkseen persoonallisuuden vaikutusta koski JAD (joint application design) kehitysmenetelmään, jossa huomioidaan käyttäjän osallistuminen projektiin joka vaiheessa~\cite{bradley1997effect}. Menetelmä ei lukeudu ketteriin menetelmiin mutta tutkimuksen tuloksia voidaan rinnastaa ketterien mentelmien toimintaan. Tarkempi kuvaus JAD:sta löytyy artikkelista~\cite{Davidson1999215}

Ketterissä menetelmissä ryhmän kokoamisen merkitys ei ole yhtä suuri verrattuna perinteisiin menetelmiin~\cite{daSilva2012}. Da Silvan ja kumppaneiden mielestä tämä voi johtua ketterien menetelmien joustavuudesta. Ketterillä ryhmillä on usein kyky muuttua ja samalla ryhmästä muotoutuu tehokkaampi. Toisaalta da Silva ja kumppanit mainitsevat että jos edellä mainittu olettamus ketteristä menetelmistä pitää paikkansa tulee ryhmän jäsenillä olla osaamista ja ymmärrystä ihmistekijöistä~\cite{daSilva2012}.


\subsection{Muutosjohtaminen ja riskienhallinta}

Ohjelmistotuotanprojektien johtamisessa suuressa roolissa ovat riskien hallinta ja muutoksen hallinta. Tutkimuksessa puhutaan usein riskien hallinnasta. Debnath ja kumppanit uudelleen kuvaavat riskien hallinnan ongelmat, muutoksen hallinnaksi ja projektin johtamiseksi~\cite{4017705}. Heidän näkemyksensä mukaan 85 prosenttia ohjelmistotuotantoprojektien riskeistä liittyvät projektin ympäristön muutoksiin~\cite{4017705}. Projektin johtajalla on mahdollisuus vaikuttaa nopealla reagoinnillaan muutoksen hyödyllisyyteen~\cite{4017705}. Debnathin ja kumppaneiden mielestä riskit eivät ole vain riskejä vaan myös mahdollisuuksia kehittää prosessia~\cite{4017705}.

Ohjelmistotuontaprojektien riskit voidaan jakaa erilaisiin luokituksiin. Fan ja Yu luokittelevat riskit organisaatio ja projektisidonnaisiin riskeihin~\cite{fan2004bbn}. Fan ja Yu esittelevät mallin, joka perustuu Bayesian Belief Networks malliin, jota projektin johtajat voivat käyttää päätöksenteon tukena riskien hallinnassa~\cite{fan2004bbn}. Bayesian Belief Networks mallia pidetään tehokkaan apuvälineenä projektin riskienhallinnassa.

Muutoksen hallinnalla ja muutosjohtamisella voidaan tarkoittaa riskien hallinnan lisäksi koko prosessiin liittyvän muutoksen hallintaa ja johtamista. Prosessin muutoksiin on vaikuttanut ketterien menetelmien yleistyminen ja projektien siirtyminen perinteisestä vesiputousmallista ketteriin menetelmiin. Päättäessään muutokseen ryhtymisestä ei johtajan pidä painottaa liiaksi projektin luontoa, projektityyppiä tai projektin aikataulua~\cite{Chow2008961}. Gou ja Seaman tutkivat syitä, jotka johtivat projektipäälliköt valitsemaan muutoksen vanhasta menetelmästä uuteen~\cite{Guo:2008:SSP:1414004.1414046}. Heidän tuloksensa osoitti että suurimmat syyt prosessin muuttamiselle ovat laatu, hinta ja aikataulu~\cite{Guo:2008:SSP:1414004.1414046}.

Ryhtyessään toteuttamaan muutosta ohjelmistotuotantoprojektien johtajat haluavat tietoja ja todisteita päätöstensä perustaksi. Tutkimuksessaan Gou ja Seaman huomasivat, että tiedot joita johtajat halusivat erosivat tiedoista joita johtajat todellisuudessa saivat ryhtyessään muutokseen~\cite{Guo:2008:SSP:1414004.1414046}. Molemmissa tilanteissa johtajan päätökseen vaikutti eniten luotettavan kollegan kokemus käyttöönotettavasta menetelmästä~\cite{Guo:2008:SSP:1414004.1414046}. Johtajat olisivat tahtoneet perustaa tietämyksensä akateemiseen tutkimukseen mutta todellisuudessa akateemisen tiedon vaikutus oli vähäisempää~\cite{Guo:2008:SSP:1414004.1414046}.

Depnath ja kumppanit esittelevät artikkelissaan ~\cite{4017705} menetelmän joka perustuu Kotterin malliin~\cite{kotter1995leading}. Esitellyssä mallissa muutoksen johtaminen jaetaan kolmeen osaan, jotka voidaan jakaa vielä pienempiin osiin. Ensimmäisessä vaiheessa johtajan tulee sulattaa tämänhetkinen tilanne. Johtajalla tulee olla syy, jonka takia tulevaisuudessa toimitaan eritavalla~\cite{4017705}. Kotterin mukaan jopa 50\% muutoksen yrityksistä epäonnistuu jo tässä vaiheessa~\cite{kotter1995leading}. Toisessa vaiheessa johtajan tulee suorittaa toimet jotka johtavat muutokseen~\cite{4017705}. Muutoksen toteuttamisessa tärkeätä on luoda visio tulevasta johon muiden henkilöiden on helppo tukeutua~\cite{4017705}. Johtaja ei voi yksinään ajaa omaa näkemystään läpi vaan johtajan tulee hankkia tukea muiden joukosta~\cite{4017705}. Mallin kolmas vaihe on muutoksen tekeminen pysyväksi~\cite{4017705}.

Scrumissa projektilla on mahdollisuus syöksyä kaaokseen. Menetelmässä kaaokseen luisuminen voidaan estää erilaisilla kontrolleilla, joita ovat tuotteen kehitysjono, parannukset, paketit, muutokset, ongelmat, riskit ja ratkaisut~\cite{schwaber1995scrum}. Näiden avulla projektiryhmä pystyy kehittämään sovellustaan sprintin aikana. Kontrollit uudelleen määritetään katselmuksessa jokaisen sprintin lopussa~\cite{schwaber1995scrum}.

Parannukset ovat osia tuotteen kehitysjonosta, jotka toteutetaan seuraavaan julkaisuun~\cite{schwaber1995scrum}. Paketit koostuvat ohjelmiston osista joita tulee muuttaa, jotta valittu ominaisuus voidaan toteuttaa~\cite{schwaber1995scrum}. Näitä osia kutsutaan muutoksiksi~\cite{schwaber1995scrum}. Ongelmat ovat tilanteita, jotka syntyvät ominaisuutta toteutettaessa. Ongelmat tulee ratkaista ennen, kuin ominaisuus saadaan toteutettua~\cite{schwaber1995scrum}. Ongelmien ratkaisut ovat yksi kontroli muoto. Riskit ovat tilanteita, jotka tarvitsevat jatkuvaa huomiota, jotta projekti voi onnistua~\cite{schwaber1995scrum}. Näiden riskien hallintaa käsiteltiin tässä luvussa aikaisemmin. 

XP:ssa on reagoitu valmiiksi muutamiin mahdollisiin riskitilanteisiin ja menetelmän kehittäjä Beck antaa ohjeistusta~\cite{796139}, kuinka toimia kyseisissä ongelmatilanteissa. Ongelmatilanteita voi syntyä kun ryhmätuotoksellisuus on yliarvioitu, asiakas ei tahdo työskennellä yhteistyössä, henkilöstöön tapahtuu muutoksia kesken projektin ja vaatimukset muuttuvat kesken kehityksen~\cite{796139}. Johtajalla on tärkeä rooli näiden ongelmien ratkaisemisessa.

Kun johtaja huomaa, että ryhmä ei saavuta luvattuja ominaisuuksia, vaan jää joka sprintissä jälkeen asetetuista tavoitteista tulee johtajan löytää mahdolliset keinot tehostaa ryhmän toimintaa. Jos johtaja ei löydä näitä keinoja tulee hänen keskustella asiakkaan kanssa tilanteesta ja päättää yhdessä, mitä siirretään tuleviin iteraatiohin.

Jos asiakas on haluton toimimaan yhteistyössä ryhmän kanssa tulee ryhmän ja ryhmän johtajan etsiä keinoja, joilla he voivat muuttaa omaa toimintaansa, jotta asiakkaan osallistuminen projektiin parantuisi. Asiakkaan kanssa tulee puhua ja etsiä syitä minkä takia asiakasta ei kiinnosta projektiin osallistuminen. XP:ssä ei ole sallittua kehitysryhmän toimiminen omien arvailujen varassa, vaan asiakkaan tulee osallistua projektia koskeviin päätöksiin~\cite{796139}.

Pariohjelmoinnin käyttö XP:ssä ehkäisee muutosten radikaalisuutta kun henkilöstö vaihtuu projektissa. Jokaista koodin osaa on ollut kehittämässä vähintään kaksi ryhmän jäsentä, joten tieto jokaisesta koodin osasta säilyy ryhmällä vaikka yksi ryhmän jäsenistä luopuisi projektiin osallistumisesta. Toinen seikka, joka helpottaa toimintaa henkilöstön vaihtuessa ovat XP:ssä suositut testit ja testaus kattavuus. Testeistä selviää mitä koodissa on tehty. Uuden henkilön liittyessä ryhmään henkilö voi harjoitella aluksi kokeneemman ja projektissa pidempään työskennelleen parina~\cite{796139}.

Ketterät menetelmät pyrkivät vastaamaan nopeisiin muutoksiin projekteissa. XP:ssä muutoksiin reagoiminen on huomioitu kehittämällä projektia pienissä osissa, joiden jälkeen projektin osiin voidaan tehdä muutoksia tarvittaessa~\cite{796139}. Iteraation pituus 2 viikkoa mahdollistaa vain pienten osien kehittämisen kerralla. Refaktorointi on XP:ssä tärkeässä roolissa, mikä helpottaa reagoimista muuttuviin tilanteisiin.

 

\section{Johtajuus ja johtajan persoonallisuus}

Ohjemistotuotantoprojekteissa johtajalla on monia mahdollisuuksia vahvistaa käytettävissä olevan ryhmänsä toimintaa. Ryhmän suoritukseen vaikuttaa johtajan persoonallisuus. Tehokaalla projektin johtajalla on kokemusta ja tietämystä käytettävistä tekniikoista ja ihmissuhdetaidoista ~\cite{McLeod:2011:FAS:1978802.1978803}. Johtajan ei tarvitse kuitenkaan olla ryhmästä henkilö, jolla on paras tuntemus käsiteltävästä aiheesta~\cite{4017705}. Johtajan persoonallisuuden vaikutuksista projektiin on tutkittu vähäisesti ohjelmistotuotantoprojekteissa~\cite{Wang:2009:PMP:1639950.1640049}.

Persoonallisuuden vaikutusta voidaan tutkia erilaisten mallien avulla ja myös yhdistelemällä menetelmiä. Tässä luvussa esitellään kaksi erilaista persoonallisuuden tutkimusmallia. Mallit ovat Five Factor Model ja persoonallisuustyyppiteoria. Samalla tutustutaan johtamistyylien vaikutukseen ohjelmistotuotantoprojektien johtamisessa. Luvun lopussa mallien ja johtamistyylien perusteella pyritään löytämään ominaisuuksia, jotka ovat johtajalle hyödyllisiä johdettaessa ohjelmistotuotantoprojektia.  



\subsection{Persoonallisuuden arviontimentelmät}

Ihmisten persoonallisuutta voidaan tutkia monilla erilaisilla menetelmillä~\cite{digman1990personality}. Yleinen menetelmä on Five Factor Model (FFM)~\cite{digman1990personality, barrick2006big}. Tämän menetelmän avulla on tutkittu persoonallisuuksien vaikutusta ohjelmistotuotantoprojektien onnistumiseen~\cite{Wang:2009:PMP:1639950.1640049}. FFM menetelmää voidaan käyttää myös arvioimaan johtajan toimintaa ja persoonallisuutta.

FFM:ssä jokainen persoonallisuuden piirre sisällytetään johonkin viidestä laajemmasta tekijästä. Viisi laajempaa tekijää, joita mallissa käytetään ovat ulospäinsuuntautuneisuus, miellyttävyys, tunnollisuus, neuroottisuus ja avoimuus uusille asiolle~\cite{barrick2006big, digman1990personality}.

Ulospäinsuuntautuneisuudesta on käytetty tutkimuksissa myös nimitystä avoimuus~\cite{digman1990personality}. Ulospäinsuuntautunut henkilö arvostaa lämpimiä ja läheisiä suhteita erilaisten ihmisten kanssa~\cite{Wang:2009:PMP:1639950.1640049}. Ulospäinsuuntautunut johtaja toimii läheisessä yhteydessä kehittäjä ryhmäänsä ja pyrkii luomaan lämpimän suhteen alaisiinsa.

Johtajan miellyttävyyteen liitetään piirteitä, jotka rakentavat ryhmän jäsenten luottamusta johtajaan. Piirteitä ovat esimerkiksi kiltteys, auttavaisuus, yhteistyökykyisyys ja huomaavaisuus~\cite{Wang:2009:PMP:1639950.1640049}. Tutkimuksissa mielyttävyydestä on käytetty myös nimitystä ystävällisyys ja vihamielisyys~\cite{digman1990personality}.

FFM:ssä tunnollisuuden tekijällä tarkoitetaan piirteitä kuten itsekuri, velvollisuudentuntoisuus ja tavoitteellisuus~\cite{Wang:2009:PMP:1639950.1640049}. Tunnollisuuteen oleellisesti liittyy henkilön halu menestyä ja toteuttaa hänelle määrätyt tehtävät. Tunnollisuudesta on käytetty tutkimuksissa myös nimitystä tahto~\cite{digman1990personality}.

Neuroottisuus kuvastaa useita negatiivisia piirteitä. Näitä piirteitä ovat esimerkiksi viha, levottomuus ja masentuneisuus~\cite{Wang:2009:PMP:1639950.1640049}. Nämä tunteet ovat usein hyvin voimakkaita. Neuroottisuudesta onkin käytetty tutkimuksissa myös lievempää nimitystä tunteiden tasapaino~\cite{digman1990personality}.

Avoimuudella FFM:ssa tarkoitetaan kykyä löytää epätavallisia ideoita, mielikuvituksellisuutta, uteliaisuutta, emootioita ja seikkailunhaluisuutta~\cite{Wang:2009:PMP:1639950.1640049}. Nimityksenä avoimuus ei täysin kuvaa piirteitä, joihin avoimuudella viitataan. Avoinmuudesta on käytetty myös nimityksiä järkevyys ja älykkyys~\cite{digman1990personality}.

Wang ja Li yhdistivät tutkimuksessaan FFM:nä Mohan Thiten kehittämään malliin jonka avulla voidaan tutkia projektin johtamisen ja projektin onnistumisen välistä sudetta ~\cite{Wang:2009:PMP:1639950.1640049}. Luomansa mallin avulla Wang ja Li tutkivat johtajan persoonallisuuden vaikutusta ohjelmistotuotantoprojektin onnistumiseen. Tutkimusten tuloksia ja FFM:n vaikutusta johtamiseen tarkastellaan myöhemmin tässä luvussa.

Persoonallisuustyyppiteoria on toinen tapa tutkia henkilöiden persoonallisuutta, jota on käytetty ohjelmistotuotantoprojekteissa. Persoonallisuus tyyppi teorian on kehittänyt Jung~\cite{jung1989psychological}. Myers ja Briggs kehittivät oman menetelmän jonka avulla voidaan arvioida henkilöitä ja sijoittaa heidät persoonallisuustyyppiteorian ulottuvuuksiin~\cite{myers1985manual}. Tästä menetelmästä on puhuttu jo aiemmin tässä tutkielmassa ryhmän kokoamisen kohdalla. Tässä vaiheessa menetelmään tutustutaan tarkemmin, jotta sen avulla voidaan arvioida johtajan persoonallisuutta.

Persoonallisuustyyppiteoriassa henkilön persoonallisuus on jaettu 4 ulottuvuuteen. Jokaisella ulottuvuudella on kaksi preferenssiä~\cite{bradley1997effect}, jotka muodostavat vastakkaisuusparin. Yhdistelemällä neljän ulottuvuuden eri preferenssejä keskenään saadaan 16 erilaista persoonallisuutta joihin henkilöt voidaan sijoittaa persoonallisuutensa mukaan.

Persoonallisuustyyppiteoriassa ulottuvuudet ovat henkilön asenne ym\-pä\-ris\-töön, henkilön tapa hankkia tietoa, päätöksenteon peruste ja elämäntyyli~\cite{bradley1997effect}. Ulottuvuuksista tiedonhankinta ja päätöksenteon peruste voidaan yhdistää toisiinsa. Yhdistämisen avulla saadaan henkilöstä tieto, minkä avulla hän tekee päätöksensä ~\cite{bradley1997effect}.

Henkilö on joko ekstrovertti(E) tai introvertti(I). Ekstroverttille on luonnollista ryhmässä toimiminen. Ekstrovertti henkilöä kiinnostaa ihmiset ja kanssakäyminen. Introvertti henkilö saa energiaa toimiessaan yksinään. Introvertti on kiinnostunut ajatuksista ja on keskittynyt omaan toimintaansa. Nämä kaksi piirrettä kuvaavat henkilön erilaista suhtautumista ympäristöönsä.

Henkilön tiedon hankinta voi perustua henkilön intuitioon(N) ja henkilö voi perustaa tietämyksensä tämän intuitionsa varaan. Toinen puoli tiedonhankinnan ulottuvuudesta on tosiasiallinen(S). Henkilö perustaa tietämyksensä koviin faktoihin. Henkilö tahtoo mahdollisimman paljon tietoa päätöksiin liittyen ennen, kuin hän tekee ratkaisun päätöksen suhteen. Tiedon hankinnan ulottuvuuden kanssa henkilön päätöksiin vaikuttaa teorian toinen ulottuvuus, päätöksenteon peruste. Henkilö perustaa päätöksensä ajatteluun(T) tai tunteisiin(F) joita hänellä on päätöksentekotilanteessa.

Teorian neljäs ulottuvuus on henkilön elämäntyyli. Ulottuvuus voidaan jakaa henkilöihin, jotka ovat harkitsevia. Harkitsevat henkilöt tahtovat elää suunniteltua elämää organisoidussa ympäristössä(J). Toinen puoli e\-lä\-män\-tyy\-lin ulottuvuutta on spontaanius(P). Henkilön haluaa elää kaaoksen rajamailla ilman turhia sääntöjä.
   

\subsection{Johtamistyylit}

Jokainen projektin johtaja on henkilönä erilainen ja johtajilla on erilaisia johtamistyylejä. Johtamistyylien vaikutusta projektiryhmän rakentamiseen ja projektin tuloksellisuuteen on tutkittu ohjelmistotuotantoprojekteissa~\cite{Dhomne:2012:ITL:2382887.2382899}.

 Dhomme ja Hall jakavat johtamistyylit kolmeen johtamistyyppiin~\cite{Dhomne:2012:ITL:2382887.2382899}. Tyypit ovat päätöksenteko-, aktiivinen johtaminen- ja persoonalliset auktoriteettiset tyylit. Päätöksentekotyylejä ovat koordinoija, itsevaltainen ja laissez-faire. Aktiivisiin tyyleihin kirjataan valmentava ja tukeva tyyli. Karisma liitetään persoonallisiin auktoriteettisiin tyyleihin. Johtamistyyleistä voidaan puhua vain päätöksentekoon liittyvillä käsitteillä, koska valmentava ja tukeva tyyli lukeutuvat koordinoijan osaksi ja karismaattinen johtaja voi kuulua mihin tahansa päätöksenteon tyyleistä~\cite{Dhomne:2012:ITL:2382887.2382899}.

Koordinoivan johtajan ryhmässä päätökset syntyvät ryhmän jäsenten päätöksistä. Johtaja pitää kuitenkin itsellään päätäntäoikeuden ongelmatilanteissa. Johtaja toimii valmentavana henkilönä, joka pyrkii kehittämään ryhmänsä jäseniä ja samalla kannustamaan heitä päätöksentekoon. Johtajan samaistuminen ryhmäläisen asemaan kuuluu koordinoivan ja tukevan johtajan tyyliin. Koordinoiva johtamistyyli voi olla aikaa tuhlaavaa, joten se ei sovellu nopeita päätöksiä vaativiin tehtäviin~\cite{Dhomne:2012:ITL:2382887.2382899}. Koordinoiva johtamistyyli ja etenkin tukeva johtamistyyli sopivat ketterien menetelmien ideologiaan~\cite{fowler2001agile}.

Itsevaltainen johtaja pitää päätökset omissa käsissään. Johtaja voi kysyä ryhmän jäseniltä mielipidettä projektiin liittyvistä kysymyksistä mutta päätökset johtaja tekee itse. Jos ohjelmistotuotantoprojektin johtajalle on annettu ryhmä, jonka tiedollinen ja taidollinen taso on paljon heikompi kuin johtajan oma tiedon ja taidon taso, on itsevaltainen johtaminen tehokasta~\cite{Dhomne:2012:ITL:2382887.2382899}. Ketterissä menetelmissä päätöksenteon vastuu on annettu kehittäjä ryhmille, joten itsevaltianen johtamistapa rikkoo ketterän manifestin periaatteita~\cite{fowler2001agile}.

Laissez-faire on johtamistyyleistä projektin kannalta usein heikoin. Johtaja ei osallistu päätöksentekoon. Johtaja jättää oman panostuksensa projektiin minimiin. Johtajat, jotka vastaavat useasta projektista saman aikaisesti ajautuvat, joissain projekteissaan tähän tyyliin oman kiinnostuksen ja ajan puutteen takia~\cite{Dhomne:2012:ITL:2382887.2382899}. Laissez-faire johtaminen voi toimia ketterässä kehityksessä koska ketterien menetelmien perustana on itsestään organisoituva kehitysryhmät~\cite{fowler2001agile}. Ryhmä tarvitsee kuitenkin johtajaa, joka pitää ryhmän oikeilla raiteilla.

Augustine ja kumppanit käyttivät tutkimuksessaan erilaista lä\-hes\-ty\-mis\-ta\-paa johtamiseen. Artikkelissaan Augustine ja kumppanit esittelevät johtamistyylin nimeltään "sopeutuva johtaminen"~\cite{Augustine:2005:APM:1101779.1101781}, jossa johtajan ei pidä kontrolloida kehitysryhmää liiaksi mutta liian vähäinen kontrollointi voi ajaa ryhmän toiminnan kaaokseen~\cite{Augustine:2005:APM:1101779.1101781}. Johtaja ymmärtää, että oppiminen ja sopeutuminen avustavat projektin etenemistä ja johtajalla on ymmärrys eri osista, jotka vaikuttavat projektin kulkuun~\cite{Augustine:2005:APM:1101779.1101781}. 





\subsection{Persoonallisuuden piirteet ja ominaisuudet}

Johtajalla on useita mahdollisuuksia vaikuttaa omalla toiminnallaan ja olemuksellaan johdettavan ryhmän toimintaan. Johtajalla on mahdollisuus vahvistaa olemuksellaan johtajan ja alaisen välistä johtaja-seuraaja sopimusta~\cite{raccoon2006leadership}. Johtajalle hyödyllisiä ominaisuuksia on löydetty ja listattu useissa tutkimuksissa~\cite{raccoon2006leadership, Wang:2009:PMP:1639950.1640049, bradley1997effect, 4017705}.

Johtajan persoonalle hyödyllisiä ominaisuuksia ovat kärsivällisyys, joustavuus, taktikointi, kommunikointitaidot, huumorintaju, auktoriteetti ja tieto~\cite{4017705}. Monet näistä piirteistä vaikuttavat toisiinsa ja samalla johtajan taidon puute yhdellä osa-alueella vaikuttaa johtajan muihin vahvoihin osa-alueisiin.

Wang ja Li löysivät tutkimuksessaan vahvistuksen olettamuksilleen FFM:n persoonallisuustekijöiden vaikutuksesta projektin onnistumiseen~\cite{Wang:2009:PMP:1639950.1640049}. FFM:n neljällä tekijällä viidestä oli positiivinen vaikutus projektin onnistumiseen. Neljän positiivisen vaikutuksen lisäksi tutkimuksessaan huomattiin että sosiaalisuudella on myös suora vaikutus projektin onnistumiseen~\cite{Wang:2009:PMP:1639950.1640049}.

Johtajan ulospäinsuuntautuneisuudella on vaikutusta projektin onnistumiseen~\cite{Wang:2009:PMP:1639950.1640049}. Ohjelmistotuotantoprojekteissa johtaja toimii asiakkaan ja muiden yhteistyötahojen kanssa yhteistyössä~\cite{McLeod:2011:FAS:1978802.1978803}, joten sosiaalisella johtajalla on näissä tilanteissa paremmat edellytykset onnistua. Ulospäinsuuntautunut johtaja osoittaa kiinnostusta ryhmänsä toimintaan, mikä voi nostaa ryhmän työskentely motivaatiota.


Miellyttävä projektijohtaja rakentaa ryhmän koheesiota. Ryhmä jonka sisäinen koheesio on parempi saavuttaa projektin tavoitteet to\-den\-nä\-köi\-sem\-min~\cite{bahli2005group}. Wang ja Li saivat samansuuntaisen päätelmän että miellyttävällä johtajalla on positiivinen vaikutus projektin onnistumiseen~\cite{Wang:2009:PMP:1639950.1640049}. Miellyttävään johtajaan on alaisten helpompi samaistua, mikä rakentaa johtaja-seuraaja -suhdetta.

Itsekuri, velvollisuudentuntoisuus ja tavoitteellisuus johtajan piirteinä vaikuttavat projektin tulokseen positiivisesti~\cite{Wang:2009:PMP:1639950.1640049}. Johtaja toimii esimiesasemassa ja samalla alaisilleen esimerkkinä, mikä mahdollisesti parantaa kehitysryhmän tuloksellisuutta. Laissez-faire johtamistyyli ei sovellu näiden luonteenpiirteiden kuvaukseen, koska laissez-faire tyylinen johtaja ei ole kiinnostunut ryhmänsä toiminnasta eikä välttämättä edes projektin lopputuloksesta. Motivoitunut johtaja stimuloi myös alaisiaan älyllisesti, mikä auttaa ryhmää löytämään ongelmiin ratkaisuja~\cite{thite2000leadership}.

Tutkimuksessaan Wang ja Li saivat selville että johtajan avoimuus ja älykkyys vaikuttavat positiivisesti projektin tuloksiini~\cite{Wang:2009:PMP:1639950.1640049}. Avoimet ihmiset soveltuvat muutosjohtamiseen koska avoimet ihmiset ovat valmiita yrittämään uusia asioita ja löytämään tehokkaampia ratkaisuja käsiteltäviin ongelmiin. Johtaja, joka ei ole avoin voi ohittaa muuttuvat tekniset vaatimukset ja mahdollisuudet. Johtamistyyleistä avoimuus voidaan liittää koordinoivaan johtamiseen, joka mahdollistaa alaistensa päätösten ja ideoiden toteutumisen.

Itsevaltainen johtaja ja laissez-faire johtaja ei huomio alaisiaan ja seuraajiaan. Koordinoiva johtaja pyrkii huomioimaan alaisensa, seuraajansa ja muut projektin onnistumiseen vaikuttavat henkilöt. Koordinoiva johtaja on usein sosiaalinen, jotta johtaja onnistuu huomioimaan kaikki projektiin osallistuvat osapuolet.

Wang ja Li huomasivat että neuroottiset luonteenpiirteet ovat negatiivisesti sidottuina projektin menestymiseen~\cite{Wang:2009:PMP:1639950.1640049}. Neuroottiset henkilöt suhtautuvat usein negatiivisesti työtään kohtaan. Neuroottinen johtaja ajautuu usein tilanteissa johtamistyyliltään Laissez-faire tyyliseen johtamiseen. Itsevaltainen johtaja voidaan kokea usein myös neuroottiseksi. Kun johtaja päättää itsenäisesti alaiset usein kokevat tämän negatiivisena ja heidät sivuuttavana. Tilanteissa johtajan luonteesta syntyy alaisille neuroottinen kuva.

Persoonallisuustyyppiteorian määrittelyllä luonnollinen johtaja on ominaisuuksiltaan ENTJ-johtaja~\cite{bradley1997effect}. Johtaja on siis ekstrovertti~\cite{bradley1997effect}. Jos johtaja olisi introvertti keskittyisi hän enemmän omaan toimintaansa kuin muiden toimintaan. Introvertti johtaja ei kykene motivoimaan ryhmäänsä. Introvertillä on hankalampaa saavuttaa johtaja-seuraaja -suhdetta alaistensan kanssa koska alaiset eivät tiedä johtajansa visiosta.

Mihin tietoon johtaja perustaa päätöksensä riippuu ongelmasta, jota projektiryhmä toteuttaa. Yksinkertaisissa ja projekteissa, joissa ei ole tarvetta kehittää uutta, johtaja perustaa päätöksensä usein kovii faktoihin~\cite{bradley1997effect}. Yksinkertaisissa projekteissa tehokas johtaja voi olla tyyliltään ESTJ. Monimutkaisissa projekteissa ja projekteissa, jossa luodaan jotain uutta johtajan päätökset perustuvat enemmän intuitioon~\cite{bradley1997effect}. Ohjelmistotuotantoprojektit ovat usein monimutkaisia joten ohjelmistotuotantoprojekteissa luonnollinen tehokas johtaja on ominaisuuksiltaan ENTJ.

Tunteillaan päättävä johtaja voi suosia eräitä ryhmän jäseniä tehdessään päätöksiä, jotka vaikuttavat projektin kulkuun. Tehokkaan johtajan tulee perustaa päätöksensä ajatteluunsa~\cite{bradley1997effect}. Tehokas johtaja tahotoo toimia organisoidussa ja hallitussa ympäristössä~\cite{bradley1997effect}. Ketterässä kehityksessä ja yleisesti ohjelmistotuotannossa toimitaan muuttuvassa ympäristössä, jossa vaatimukset muuttuvat. Johtajan kannalta on hyödyllistä että ryhmän toimintaa kontrolloi jokin kehitysmenetelmä kuten XP tai scrum, jotta saavutetaan organisoitu ympäristö.

Johtajan karisma nousee esiin usein tutkimuksissa puhuttaessa tehokkaasta johtamisesta~\cite{Dhomne:2012:ITL:2382887.2382899, thite2000leadership}. Karismaattinen johtaja kykenee esittämään alaisilleen ideansa ja uskomuksensa niin että alaisten on helpompi samaistua arvioihin~\cite{thite2000leadership}. Alaisen samaistuessa johtajan arvoihin syntyy johtajan ja alaisen välille johtaja-seuraaja -suhde~\cite{raccoon2006leadership}.

Johtajan persoonallisuudella ja johtamistavalla on suuri vaikutus ohjelmistotuotantoprojekteihin~\cite{Wang:2009:PMP:1639950.1640049}. Tutkielmassa on esitelty toimivia piirteitä johtajalle. Johtamisessa on tärkeää muistaa että johtaminen on tilanteeseen sidottua. On mahdotonta löytää yhtä johtajatyyppiä, joka soveltuu jokaiseen tilanteeseen~\cite{thite2000leadership}.

  



\section{Yhteenveto}

Ohjelmistotuotantoprojekteissa toimitaan muuttuvassa ja monimutkaisessa ympäristössä. Ketterien menetelmien kehittyminen on osaltaan helpottanut reagoimista muutoksiin ja toimimista monimutkaisissa tilanteissa. Ketterät menetelmät eivät yksinään riitä ratkaisuksi projektien ongelmien ratkaisuun vaan projektiryhmän sisällä tarvitaan henkilöiden toimintaa, joka auttaa projektiryhmää onnistumaan.

Tässä tutkielmassa on selvitetty johtamiseen liittyviä erilaisia osa-alueita, joiden huomiointi ohjelmistotuotantoprojekteissa on tärkeää. Suurin jaottelu on tehty johtajuuden ja hallinnan välillä. Johtajuus on osa henkilön persoonaa. Johtajuus on aina tilanteeseen sidottua. Johtajuus jaetaan kahteen alalajiin, transformaatiojohtamiseen ja transaktiojohtamiseen. Hallintaan liitetään projektin kontrollointiin liittyvä toiminta, joka ei ole henkilöihin sidottua. Ylemmän hallinnon johtaminen luetaan usein hallinnan piiriiin.

Ketterissä menetelmissä johtaminen on jaettu kehitysryhmän sisällä. Jokaisen ryhmän jäsenen tulee johtaa ryhmän toimintaa omalla tekemisellään. Johtaminen ei ole sidottu tiettyyn rooliin tai tehtävään. Jokaisen vastuulla oleva johtaminen tehostaa ryhmän toimintaa ja tuo esiin erilaisia näkemyksiä tilanteisiin.

Vaikutukset ohjelmistotuotantoprojektien onnistumiseen alkavat projektin suunnittelusta ja projektiryhmän kokoamisesta. Projektiryhmän kokoamisessa johtajan on tärkeää huomioida henkilöiden ominaisuudet ja persoonallisuudet, mitkä hän valitsee projektiin.Tehokas projektiryhmä on heterogeeninen eli ryhmä koostuu erilaisista persoonista ja erilaisia taitoja omaavista henkilöistä. Persoonallisuuden piirteiden tarkastelemiseen johtajalla on useita työkaluja, joista tässä tutkielmassa esiteltiin I Opt ja MBTI.

Projektiryhmän kokoamiseen vaikuttaa oleellisesti ylemmän hallinnon tarjoamat resurssit. Suuremmissa organisaatiossa projektit joutuvat usein kilpailemaan parhaista kyvyistä organisaation sisällä. Ylempi hallinto vaikuttaa sallittuihin kehitysmenetelmiin. Projektiryhmän johtajan tulee huomioida hänen vaikutusmahdollisuutensa projektissa käytettävään kehitysmenetelmään, jotta hän osaa toimia mahdollisimman hyvin projektin eri vaiheissa. Scrumissa scrummaster vaikuttaa scrumtiimin toimintaan scrumin periaatteiden kautta.

Ohjelmistot ovat alttiita muutoksille. Suuri osa johtajan tehtävästä koostuu muutoksien synnyttämien riskien hallinnasta. Nopealla toiminnalla johtajalla on mahdollisuus kääntää riskit mahdollisuuksiksi kehittää prosessia ja parantaa ryhmän motivaatiota.

Johtajan persoonallisuudella on vaikutusta johtamiseen ja johtamistyyliin. Erilaisten persoonallisuuksien vaikutusta ohjelmistotuotantoprojektien johtamiseen on tutkittu vähäisesti. Tutkimuksissa on löydetty erilaisia tehokkaita persoonallisuuden yhdistelmiä, jotka tehostavat johtamista ohjelmistotuotantoprojekteissa. Johtajuuden tilannesidonnaisuus estää kuitenkin löytämästä yhtä tai kahta optimaalista johtajan persoonallisuuden kuvausta.

Tutkielmassa esitellyt hyödylliset johtamisen piirteet perustuvat tutkimuksiin, joissa on tutkittu johtajan persoonallisuuden vaikutusta ohjelmistotuotantoprojektien onnistumiseen. Projektin onnistuminen ei kuitenkaan aina kerro huonosta tai hyvästä johtamisesta. Johtaja voi onnistua tehtävässään vaikka ohjelmistotuotantoprojekti ei saavuta tavoitteitaan. Näissä tilanteissa organisaation sisäinen toiminta on kehittynyt johtajan toimien ansiosta tai ryhmän jäsenet ovat saaneet tärkeää osaamista projektin aikana.

Eräs mielenkiintoinen jatkotutkimuksen aihe on kuinka johtamista opetetaan ohjelmistoalalla. Kun johtaminen on nykyään kaikkien vastuulla, voidaan pohtia, tulisiko oppilaitosten opettaa johtamista IT-alan opiskelijoille. Johtamisen opettamisesta kehitysryhmälle ei ole juurikaan tutkimustuloksia. Mikä on johtamisosaamisen vaikutus projektin tuloksellisuuteen?

\newpage 

Johtajuuden ja persoonallisuuden tutkiminen on osoittautunut hankalaksi. Psykologian tutkimuksen näkökulmasta johtajuuden tutkimukseen löytyy useita eroavia näkökantoja~\cite{haslam2011}. Johtajuuden psykologisella tutkimuksella on vaikutusta myös ohjelmistotuotantoprojektien johtajuuden tutkimukseen. Johtajuus ja sen tutkimus ovat koko ajan kehittymässä, joten tutkimusten tekeminen johtajuudesta ohjelmistotuotantoprojekteissa vaikuttaa ohjelmistotuotantoprojektien toimintaan ja tehokkuuteen.   

   

\newpage 



 











% --- Back matter ---
%

% bibtex is used to generate the bibliography. The babplain style
% will generate numeric references (e.g. [1]) appropriate for theoretical
% computer science. If you need alphanumeric references (e.g [Tur90]), use
%
% \bibliographystyle{babalpha}
%
% instead.

\bibliographystyle{babplain}
\bibliography{references-fi}


\end{document}
