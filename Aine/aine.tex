% --- Template for thesis / report with tktltiki2 class ---

\documentclass[finnish]{tktltiki2}

% tktltiki2 automatically loads babel, so you can simply
% give the language parameter (e.g. finnish, swedish, english, british) as
% a parameter for the class: \documentclass[finnish]{tktltiki2}.
% The information on title and abstract is generated automatically depending on
% the language, see below if you need to change any of these manually.
% 
% Class options:
% - grading                 -- Print labels for grading information on the front page.
% - disablelastpagecounter  -- Disables the automatic generation of page number information
%                              in the abstract. See also \numberofpagesinformation{} command below.
%
% The class also respects the following options of article class:
%   10pt, 11pt, 12pt, final, draft, oneside, twoside,
%   openright, openany, onecolumn, twocolumn, leqno, fleqn
%
% The default font size is 11pt. The paper size used is A4, other sizes are not supported.
%
% rubber: module pdftex

% --- General packages ---

\usepackage[utf8]{inputenc}
\usepackage{lmodern}
\usepackage{microtype}
\usepackage{amsfonts,amsmath,amssymb,amsthm,booktabs,color,enumitem,graphicx}
\usepackage[pdftex,hidelinks]{hyperref}

% Automatically set the PDF metadata fields
\makeatletter
\AtBeginDocument{\hypersetup{pdftitle = {\@title}, pdfauthor = {\@author}}}
\makeatother

% --- Language-related settings ---
%
% these should be modified according to your language

% babelbib for non-english bibliography using bibtex
\usepackage[fixlanguage]{babelbib}
\selectbiblanguage{finnish}

% add bibliography to the table of contents
\usepackage[nottoc,numbib]{tocbibind}
% tocbibind renames the bibliography, use the following to change it back
\settocbibname{Lähteet}

% --- Theorem environment definitions ---

\newtheorem{lau}{Lause}
\newtheorem{lem}[lau]{Lemma}
\newtheorem{kor}[lau]{Korollaari}

\theoremstyle{definition}
\newtheorem{maar}[lau]{Määritelmä}
\newtheorem{ong}{Ongelma}
\newtheorem{alg}[lau]{Algoritmi}
\newtheorem{esim}[lau]{Esimerkki}

\theoremstyle{remark}
\newtheorem*{huom}{Huomautus}


% --- tktltiki2 options ---
%
% The following commands define the information used to generate title and
% abstract pages. The following entries should be always specified:

\title{Aine: Johtaminen ja johtajuus, rooli ja vaikutus ohjelmistotuotantoprojekteissa: ketterät mentelmät ja avoin kehitys}
\author{Mika Kivi}
\date{\today}
\level{Aine}
\abstract{Tiivistelmä.}

% The following can be used to specify keywords and classification of the paper:

\keywords{johtaminen, ohjelmistotuotanprojektin johtaminen, muutos johtaminen}
\classification{} % classification according to ACM Computing Classification System (http://www.acm.org/about/class/)
                  % This is probably mostly relevant for computer scientists

% If the automatic page number counting is not working as desired in your case,
% uncomment the following to manually set the number of pages displayed in the abstract page:
%
% \numberofpagesinformation{16 sivua + 10 sivua liitteissä}
%
% If you are not a computer scientist, you will want to uncomment the following by hand and specify
% your department, faculty and subject by hand:
%
% \faculty{Matemaattis-luonnontieteellinen}
% \department{Tietojenkäsittelytieteen laitos}
% \subject{Tietojenkäsittelytiede}
%
% If you are not from the University of Helsinki, then you will most likely want to set these also:
%
% \university{Helsingin Yliopisto}
% \universitylong{HELSINGIN YLIOPISTO --- HELSINGFORS UNIVERSITET --- UNIVERSITY OF HELSINKI} % displayed on the top of the abstract page
% \city{Helsinki}
%


\begin{document}

% --- Front matter ---

\maketitle        % title page
\makeabstract     % abstract page

\tableofcontents  % table of contents
\newpage          % clear page after the table of contents


% --- Main matter ---

\section{Johdanto}

Ohjelmistojen merkitys ihmisten jokapäiväisessä elämässä on kasvanut. Samalla ohjelmistoja on kehitetty sovellus alueille, joissa virheen sattumisella voi olla kohtalokkaat seuraukset.

Ohjelmistotuotantoprojekti epäonnistuvat usein saavuttamaan sille asetetut tavoitteet. Syitä ohjelmistotuotantoprojektien epäonnistumiselle on pyritty tutkimaan. Syyt liittyvät usein ohjelmiston laatuun tai aikataulun ja budjetin ylityksiin.

Ohjelmistotuotanprojektien onnistumiseen johtavia tekijöitä pyritty selvittämääm tutkimuksilla. Ohjelmistotuotanto on ihmislähtöistä toimintaa, jossa ihmiset tekevät ohjelmia ihmisille~\cite{Wang:2010:PPP:1810295.1810302}. Ohjelmistotuotantoprojektin tuloksiin vaikuttavat ihmiset, toimintatavat, kehitysprosessi ja projektin ympäristö. 

Tässä aineessa pyrin selvittämään ihmisten ja toimintatapojen vaikutusta projektin lopputuloksiin. Aine keskittyy johtamisen ja johtajuuden vaikutuksiin ohjelmistotuotantoprojektiin. Tavoitteena on myös löytää keinoja joiden avulla projektin johtajalla on mahdollisuus vaikuttaa projektin lopputulokseen. Ohjelmistotuotantoprojektin johtamista tutkitaan tarkemmin ketterien menetelmien ja avoimen kehityksen näkökulmista.

Aine jakautuu viteen osaan. Ensimmäisessä osassa esitellään käsiteltävät kehitys metodologia. Toinen osa keskittyy ohjelmistotuotanprojektin johtamiseen yleisellä tasolla. Seuraavat kaksi osiota syvennytään johtamiseen ketterissä menetelmissä sekä avoimessa ohjelmistokehityksessä. Viimeisessä osassa etsitään keinoja kuinka johtaja ja johtajan persoonallisuus voivat vaikuttaa ohjelmistotuotantoprojektin tuloksiin.
  



\section{Ohjelmsitotuotantoprojektin kehitys menetelmät}

Ohjelmistojen tuottamiseen on kehittynyt erialisia metodologiota kehittään ohjelmistoja. Tässä aineessa käsittelen tarkemmin johtamista ketterien menetelmien ja avoimen ohjelmistokehityksen näkökulmasta. Tässä osassa esittelen nämä kaksi metdologiaa.

\subsection{Ketterät menetelmät}

Ketterillä menetelmillä tarkoitetaan kehitys kehyksiä joita ohjelmistotuotantoprojekteissa voidaan käyttää. Ketterät menetelmät perustavat ketterään manifestiin (Tähän tulee manifesti lähteeksi miten sitä saa käyttä). Ketterein menetelmien avulla pyritään luomaan kehittäjille työrauha. Tärkeintä ketterissä menetelmissä on aikainen ja jatkuva tuotteen näyttäminen asiakkaalle. Tämän avaulla saavutetaan paras asiakastyytyväisyys. Ketteriä menetelmiä on arvosteltu aikataulun ja budjetin ylittämisestä~\cite{Guo:2008:SSP:1414004.1414046}.


\subsection{Avoin ohjelmistokehitys}

Avoimessa ohjelmistotuotannossa ohjelmistoa kehittävä yhteisö on pääsääntöisesti hajautunut, eivätkä ihmiset tunne toisiaan. Ohjelmiston kehittäminen perustuu vapaaehtoisuuteen, joten kehittäjiä kiinnosta jokin muu kuin raha. Avoin ohjelmisto kehitys koetaan tärkeäksi, josta kertoo myös monien avointen projektien rooli tämän päivän maailmassa. ~\cite{Li:2006:MOS:1125170.1125182}

\section{Johtaminen ja johtajuus ohjelmistotuotannossa}

Ohjelmistotuotannossa johtamisella voidaan tarkoittaa useita erilaisia johtamisen tasoja. Tässä osassa selvitetään kuinka erilaiset johtamisen tasot vaikuttavat ohjelmistotuotantoprojektiin. Aluksi puhutaan johtamisesta yleisesti ja siitä missä johtamista esiintyy. Tämän jälkeen käsitellään ylemmän hallinon johtamisen vaikutusta ohejlmistotuotanprojektin toimintaan. Tämän jälkeen siirrytään itse projektin johtamiseen. Tämän osan lopussa käydään läpi johtajan persoonallisuuden vaikutusta johtamiseen.

\subsection{Yleinen johtaminen}

Tässä tullaan avamaan johtamisen käsitettä yleisesti?

\subsection{Ylemmän hallinon vaikutus}

Kuinka ylempi organisaatio vaikuttaa projektin johtamiseen ja johtajaan?

\subsection{Kehitysryhmän johtaminen}

Ohjelmistotuotantoprojekteissa johtajan tehtävänä on motivoida muita. Johtajan tehtäviin voidaan lukea mm. organisointi, innovointi, arvionti, viiemistelijä  ja yleisellä tasolla tuotoksesta vastaaminen. ~\cite{4017705}

\subsection{Johtajan persoonallisuus}

Millainen johtajan persoonnalisuuden tulisi olla?

Ohjemistotuotantoprojekteissa johtajalla on monia mahdollisuuksia vahvistaa käytettävissä olevan ryhmänsä toimintaa. Ryhmä suoritukseen vaikkuttaa johtajan persoona. Johtajan persoonalle hyödyllisiä ominaisuuksia ovat kärsivällisyys, joustavuus, taktikointi, kommunikointitiadot, huumorintaju, auktoriteetti ja tieto. Monet näistä piirteistä vaikuttavat toisiinsa. Johtajan on helmpompaa saavuttaa auktoriteetin asema ryhmssään, jos hänellä on tietämystä aihealueesta. Johtajan ei tarvitse kuitenkaan olla ryhmästä se henkilö, jolla on paras tuntemus käsiteltävästä asiasta.~\cite{4017705} 

\section{Johtajan rooli ketterissä menetelmissä}

Ketterissä menetelmissä projektin johtajan rooli eroo suurelta osin yleisestä johtamisen käsitteestä. Tässä luvussa käsitellään projektin johtamista ketterissä menetelmissä. Johtajan roolia tarkastellaan tarkemmin kahdessa ketterässä menetelmässä. 

\subsection{Scrum}

\subsection{Extreme programing(XP)}

\section{Jothamisen tehtävät avoimessa ohjelmisto kehityksessä}

Avoimessa ohjelmisto kehityksessä johtajan rooli on monimutkainen. Tässä luvussa selvitetään johtajan tehtävää avoimessa ohjelmisto kehityksessä. Pyrin myös löytämään ratkaisu keinoja joilla avoimen ohjelmisto prosessin johtaja voi tehostaa prosessin toimintaa.

\subsection{Johtamis käytänteet}

Avoimessa ohjelmistokehityksessä johtajaa ei usein määräydä ylemmän organisaation määräyksestä. Johtavan henkilön asemaan voidaan päästä kahdella tavalla. Kun edellinen projektin johtaja päättää jättää projektin kehittämisen hän voi määrätä tai ehdottaa itselleen jatkaa. Toisessa tilanteessa yhteisö, joka ohjemistoa kehittää valitsee jäsenistöstään iteselleen johtavan henkilön. ~\cite{Li:2006:MOS:1125170.1125182}

\subsection{Johtajan vaikutus prosessiin}

Johtajan tärkeimmäksi tehtäväksi avoimessa ohjelmistokehityksessä  voidaan sanoa yhteisön motvoinnin. Kuinka saada ihmiset kontribuoimaan.~\cite{Li:2006:MOS:1125170.1125182}


\section{Johtajan vaikutus ohjelmistotuotantoprojekteissa}

Johtaja mahdollisuudet vaikuttaa ohjelmistotuotantoprosessin lopputulokssen on monet. Tässä luvussa käsitellään tapoja joilla johtaja voi vaikuttaa ohjelmisotuotanprojektin lopputuloksiin. Luvussa käsitellään myös aiheita, joita projektin johtajan tulee ottaa huomioon kehityksen aikan, jotta projekti voi onnistua. 

Johtajan persoonan vaikutusta voidaan tutkia Five Factor personal mallin avulla ja Mohan Thiten mallin avulla~\cite{Wang:2009:PMP:1639950.1640049}. 


\section{Yhteenveto}







 






Esimerkkilause ja lähdeviite~\cite{McLeod:2011:FAS:1978802.1978803}
Esimerkkilause ja lähdeviite~\cite{Guo:2008:SSP:1414004.1414046}.
Esimerkkilause ja lähdeviite~\cite{Luther:2008:LOC:1460563.1460619}
Esimerkkilause ja lähdeviite~\cite{Zhang:2011:ECL:2047594.2047666}
Esimerkkilause ja lähdeviite~\cite{Dhomne:2012:ITL:2382887.2382899}
Esimerkkilause ja lähdeviite~\cite{1385637}
Esimerkkilause ja lähdeviite~\cite{bahli2005group}
Esimerkkilause ja lähdeviite~\cite{Augustine:2005:APM:1101779.1101781}
Esimerkkilause ja lähdeviite~\cite{Chow2008961}






% --- Back matter ---
%
% bibtex is used to generate the bibliography. The babplain style
% will generate numeric references (e.g. [1]) appropriate for theoretical
% computer science. If you need alphanumeric references (e.g [Tur90]), use
%
% \bibliographystyle{babalpha}
%
% instead.

\bibliographystyle{babplain}
\bibliography{references-fi}


\end{document}