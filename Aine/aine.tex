% --- Template for thesis / report with tktltiki2 class ---

\documentclass[finnish]{tktltiki2}

% tktltiki2 automatically loads babel, so you can simply
% give the language parameter (e.g. finnish, swedish, english, british) as
% a parameter for the class: \documentclass[finnish]{tktltiki2}.
% The information on title and abstract is generated automatically depending on
% the language, see below if you need to change any of these manually.
% 
% Class options:
% - grading                 -- Print labels for grading information on the front page.
% - disablelastpagecounter  -- Disables the automatic generation of page number information
%                              in the abstract. See also \numberofpagesinformation{} command below.
%
% The class also respects the following options of article class:
%   10pt, 11pt, 12pt, final, draft, oneside, twoside,
%   openright, openany, onecolumn, twocolumn, leqno, fleqn
%
% The default font size is 11pt. The paper size used is A4, other sizes are not supported.
%
% rubber: module pdftex

% --- General packages ---

\usepackage[utf8]{inputenc}
\usepackage{lmodern}
\usepackage{microtype}
\usepackage{amsfonts,amsmath,amssymb,amsthm,booktabs,color,enumitem,graphicx}
\usepackage[pdftex,hidelinks]{hyperref}

% Automatically set the PDF metadata fields
\makeatletter
\AtBeginDocument{\hypersetup{pdftitle = {\@title}, pdfauthor = {\@author}}}
\makeatother

% --- Language-related settings ---
%
% these should be modified according to your language

% babelbib for non-english bibliography using bibtex
\usepackage[fixlanguage]{babelbib}
\selectbiblanguage{finnish}

% add bibliography to the table of contents
\usepackage[nottoc,numbib]{tocbibind}
% tocbibind renames the bibliography, use the following to change it back
\settocbibname{Lähteet}

% --- Theorem environment definitions ---

\newtheorem{lau}{Lause}
\newtheorem{lem}[lau]{Lemma}
\newtheorem{kor}[lau]{Korollaari}

\theoremstyle{definition}
\newtheorem{maar}[lau]{Määritelmä}
\newtheorem{ong}{Ongelma}
\newtheorem{alg}[lau]{Algoritmi}
\newtheorem{esim}[lau]{Esimerkki}

\theoremstyle{remark}
\newtheorem*{huom}{Huomautus}


% --- tktltiki2 options ---
%
% The following commands define the information used to generate title and
% abstract pages. The following entries should be always specified:

\title{Aine: Johtaminen ja johtajuus, rooli ja vaikutus ohjelmistotuotantoprojekteissa: ketterät mentelmät}
\author{Mika Kivi}
\date{\today}
\level{Aine}
\abstract{Tiivistelmä.}

% The following can be used to specify keywords and classification of the paper:

\keywords{johtaminen, ohjelmistotuotanprojektin johtaminen, muutos johtaminen}
\classification{} % classification according to ACM Computing Classification System (http://www.acm.org/about/class/)
                  % This is probably mostly relevant for computer scientists

% If the automatic page number counting is not working as desired in your case,
% uncomment the following to manually set the number of pages displayed in the abstract page:
%
% \numberofpagesinformation{16 sivua + 10 sivua liitteissä}
%
% If you are not a computer scientist, you will want to uncomment the following by hand and specify
% your department, faculty and subject by hand:
%
% \faculty{Matemaattis-luonnontieteellinen}
% \department{Tietojenkäsittelytieteen laitos}
% \subject{Tietojenkäsittelytiede}
%
% If you are not from the University of Helsinki, then you will most likely want to set these also:
%
% \university{Helsingin Yliopisto}
% \universitylong{HELSINGIN YLIOPISTO --- HELSINGFORS UNIVERSITET --- UNIVERSITY OF HELSINKI} % displayed on the top of the abstract page
% \city{Helsinki}
%


\begin{document}

% --- Front matter ---

\maketitle        % title page
\makeabstract     % abstract page

\tableofcontents  % table of contents
\newpage          % clear page after the table of contents


% --- Main matter ---

\section{Johdanto(Sisältö muuttui poistetaan tuo avoin ohjelmisto kehitys koska turhan paljon asiaa ja aineeseen ei tule enää paljoa agilestakaan koska sitä käsitellään tarkemmin sitten lopullisessa työssä)}

Tietokoneet ja tietokoneiden kaltaiset laitteet ovat yleistyneet ihmisten käytössä. Tietokoneita ja laitteita, jotka toimivat ohjelmistoilla löytyy ihmisten kodeista ja työpaikoilta. Näihin laitteisiin kehitetään jatkuvasti uusia ohjelmistoja ja vanhoja ohjelmistoja kehitetään paremmiksi. Ohjelmistojen kehittäminen on kuitenkin monimutkainen prosessi, jonka onnistumiseen vaikuttavat monet tekijät.

Ohjelmistotuotanto on ihmislähtöistä toimintaa, jossa ihmiset tekevät ohjelmistoja ihmisille~\cite{Wang:2010:PPP:1810295.1810302}(kaivettava alkup?). Ihmisillä on siis suuri vaikutus projektien onnistumiseen ja epäonnistumiseen. Muita projektin tulokseen vaikuttavia tekijöitä ovat projektin toimintavat, käytettävä kehitysprosessi ja ympäristö, jossa projektia toteutetaan~\cite{McLeod:2011:FAS:1978802.1978803}. 


Tässä aineessa pyrin selvittämään ihmisten ja toimintatapojen vaikutusta projektin lopputuloksiin. Aine keskittyy johtamisen ja johtajuuden vaikutuksiin ohjelmistotuotantoprojektiin. Tavoitteena on myös löytää keinoja joiden avulla projektin johtajalla on mahdollisuus vaikuttaa projektin lopputulokseen. Ohjelmistotuotantoprojektin johtamista tutkitaan tarkemmin ketterien menetelmien ja avoimen kehityksen näkökulmista.

Johtamista käsitellään eri tasoilla. Käsiteltävänä on kehitysryhmän sisäinen ja ulkoinen johtaminen. Aineessa luodaan myös katsausta ylemmän organisaation johtamisen vaikutuksesta ryhmän toimintaan. 

Aine jakautuu viteen osaan. Ensimmäisessä osassa esitellään käsiteltävät kehitys menetelmät. Toinen osa keskittyy ohjelmistotuotanprojektin johtamiseen yleisellä tasolla. Seuraavat kaksi osiota syvennytään johtamiseen ketterissä menetelmissä sekä avoimessa ohjelmistokehityksessä. Viimeisessä osassa etsitään keinoja kuinka johtaja ja johtajan persoonallisuus voivat vaikuttaa ohjelmistotuotantoprojektin tuloksiin.
  



\section{Ohjelmsitotuotantoprojektin kehitys menetelmät}

Ohjelmistojentuottamiseen on kehittynyt erialisia kehitys menetelmiä. Tässä aineessa käsittellään tarkemmin johtamista ketterien menetelmien ja avoimen ohjelmistokehityksen näkökulmasta. Tässä luvussa esitellään kyseisten menetelmien ideaologioita ja persukäsitteistöä. Ketteristä menetelmstä esitellään yksittäisiä menetelmiä ja avoimesta kehityksestä esitellään tunnetuja projekteja.

\subsection{Ketterät menetelmät}



Ketterät menetelmät (agile methodologies) on kokoelma erilaisia kehitys menetelmiä, joiden ideologia perustuu ketterän manifestin ajatuksiin~\cite{fowler2001agile}. Menetelmien tarkoituksena on pyrkiä mahdollisimman hyvään asiakas tyytyväisyyteen. Tämän tavoitteen saavuttamiseksi ohjelmistosta pyritään tarjoamaan asiakkaalle toimiva versio jatkuvasti kehityksen aikana~\cite{fowler2001agile}. Ketterät menetelmät perustuvat iteratiiviseen kehitykseen, jossa kehitystä tehdään sykäyksissä. Sykäyksen jälkeen tarkastellaan aikaan saannosta ja päätetään mitä seuraavassa sykäyksessä tehdään. 

Manifesti os syntynyt vuonna 2001 17 tunnetun ohjelmistoalanammattilaisen tapaamisessa. Ryhmän jäsenten oli tarkoitus löytää yheteinen käsitys uusista menetelmistä, joita oli ohjelmistojen kehitykseen syntynyt. Ketteriksi menetelmiksi voidaan lukea muunmuassa Extreme programing(XP), SCRUM, Crystal  ja Feature-Driven development~\cite{fowler2001agile}. Tässä aineessa ketteristä menetelmistä käsitellään tarkemmin johtamista XP ja SCRUM kehyksissä. 

Ketteriä menetelmien toivottiin ratkaisevan ohjelmistotuotantoprojektien ongelmat. Manifestin kirjoittajat kuitenkin korostava että he eivät ole löytäneet täydellistä ratkaisua ohjelmistojen kehityksen ongelmiin. Ketteriä menetelmiä on arvosteltu aikatualun ja budjetin ylittämisestä~\cite{Guo:2008:SSP:1414004.1414046}.   

Ketterät menetelmät korostavat ihmisten välistä vuorovaikutusta. Vuorovaikutuksen korostaminen vaikuttaa oleellisesti johtamiseen. Manifesti kehoittaa kasvokkain tapahtuvaan vuorovaikutukseen. Vuorovaikutusta asiakkaan ja kehitysryhmän välillä korostetaan.

 SCRUM menetelmän periaateet esiteltiin ensimmäisen kerran 1986. Periaateet esitteli Hirotaka Takeuchi ja Ikujiro Nonaka kirjassaan "The New New Product Development Game"~\cite{nonaka1986new}. SCRUM perustuu kehitysryhmän itsehalllinnalle. Kehitys ryhmällä ei ole varsinaista johtaa. Päätökset ryhmä tekee yhdessä. SCRUM:ssa tärkeitä rooleja ovat SCRUM master, SCRUM team ja product owner ~\cite{schwaber2002agile}. 

Extreme Programing (XP) on kehittänyt Kenet Beck joka artikkelissaan "Embracing change with extreme programming" menetelmän ideologiaa~\cite{796139}. XP:ssä säännökset toiminnalle ovat tarkemmat kuin SCRUM:ssa. Projektin työskentely tila on XP:ssa avoin, jossa keskellä on tietokoneet, joilla ohjelmointi toteutetaan. Ohjelmointi toteutetaan pari ohjlemointina. Jokaisella parilla on vain yksi näyttö, näppäimistö ja hiiri. Ohjelmen osien kehitystö ei ole osoitettu tietyille henkilöille vaan jokaisella ryhmänjäsenellä on oikeus muuttaa ohjelman kaikkia osa-alueita. XP:ssä testauksen merkitystä korostetaan. Asiakas osallistuu projektiin täyspäiväisesti eli on osa kehitys ryhmään~\cite{796139}.  



\subsection{Avoin ohjelmistokehitys (Tämä poistuu ja aine käsittelee enää vain ketteriä)}

Avoimet ohjelmistot ovat ohjelmia joita kehittävät ryhmät, jotka koostuvat vapaaehtoisesista ihmisistä joille ei makseta palkkaa ohjelmiston kehittämisestä. Ryhmien jäsenistö on maantieteellisesti hajaantunutta. Kommunikaatio ja osallistuminen kehitystoimintaan tapahtuu internetin välityksellä.

Avoin ohjelmisto kehitys koetaan tärkeäksi, josta kertoo myös monien avointen projektien rooli tämän päivän maailmassa~\cite{Li:2006:MOS:1125170.1125182}. Tunnettuja avoimen ohjelmistokehityksen tuotteita ovat Linux, Apache, Mozilla Firefox, Netbeans ja Eclipse. 

Avoimille ohjelmistoille löytyy usein kaupallinen vastine, jota vastaan avoinohjelmisto kilpailee~\cite{Luther:2008:LOC:1460563.1460619}(Alkuperäinen tarvitaan). Linux, joka on avoin käyttöjärjestelmä on vastine Microsoftin Windows käyttöjärjestelmälle ja Netbeans sekä Eclipse ovat Microsoftin Visual studion avoimia vastikkeita. Avoimesti kehitetyt ohjelmistot ovat pääosin ilmaisia joten niillä on hinnan perusteella etulyönti asema kaupallisiin kilpailijoihinsa nähden. 

Avoimen ohjelmitokehitysprojektin johtamista tarkastellaan tässä aineessa kolmen eri projektin näkökulmasta. Käsiteltäviä projekteja ovat Linux, Apache ja Netbeans. Käsiteltävien projektien ohella luodaan katsasus verkosssa toimivien kehitys yhteisöiden johtamisesta. 


Linux käyttöjärjestelmän kehittymiseen on vaikuttanut ihmisten tyytymättömyys dos ja windows käyttöjärjestelmiin. Linux kehittämisen aloitti suomalainen Linus Torvalds. Torvaldsin avuksi Linuxin ympärille kehittyi yhteisö, joka kehittää ohjelmistoa. Linuxin ensimmäinen versio tarjottiin minix yhteisön jäsenten kokeiltavaksi vuonna 1991. Linuxin ympärillä toimivan kehittäjä yhteisön koko on kasvanut ja se on luultavasti suurin ohjelmistoa kehittävä ryhmä maailmassa niin kaupallisella kuin avoimen kehittämisen puolella~\cite{conlon2007examination}. 

Apache on yleisesti käytössä oleve web-palvelin. Sen kehittämisen aloitta Rob Hover. Tuolloin Apache kulki nimelle Httpd. Hoverin lopettaessaan kehittämisen kehitys vastuu jäi webmastereiden vastuulle, jotka perustivat keskus hakemisto johon päivitykset tehtiin. Apache kirjoitettiin Httpd pohjalta ja nykyään se koostuu enään pieneltä osin esi-isäänsä. Apache yheitsön koko oli vuonna 2007 noin 800 kehittäjää~\cite{conlon2007examination}.

Netbeans on kehitysympäristö jota kehittää ja ylläpitää Netbeans.org-yhteisö. Netbeanssin kehitys on aloitettu vuonna 1996. Netbeans oli alunperin opiskelija lähtöinen projekti,  jonka  Sun julkaisi avoimeen kehitykseen~\cite{1385637}. Kehittäjät osallistuvat yhteisön toimintaan sähköpostilistoilla, korjaamalla ohjelmistoa, päivittämällä nettsivuja tai tekemällä käännöksiä ohjelmsta. 

\section{Johtaminen ja johtajuus ohjelmistotuotannossa}

Ohjelmistotuotannossa johtamisella voidaan tarkoittaa useita erilaisia johtamisen tasoja. Tässä osassa selvitetään kuinka erilaiset johtamisen tasot vaikuttavat ohjelmistotuotantoprojektiin. Aluksi puhutaan johtamisesta yleisesti ja siitä missä johtamista esiintyy. Tämän jälkeen käsitellään ylemmän hallinon johtamisen vaikutusta ohejlmistotuotanprojektin toimintaan. Tämän jälkeen siirrytään itse projektin johtamiseen. Tämän luvun lopussa käydään läpi johtajan persoonallisuuden vaikutusta johtamiseen.

\subsection{Johtamistyylit}

Jokainen projektin johtaja on henkilönä erilainen ja johtajilla on erilaiset johtamistyylit. Johtamistyylejä vaikutusta projektiryhmän rakentamiseen ja projektin tuloksellisuuteen on tutkittu. 

Artikkelissaan ~\cite{Dhomne:2012:ITL:2382887.2382899} Dhomme ja Hall jakavat johtamistyylit päätöksenteko-, aktiivinen johtaminen- ja persoonalliset auktoriteettiset tyylit. Päätöksenteko tyylejä ovat koordinoija, autoritaarinen ja laissez-faire. Aktiivisiin tyyleihin kirjataan valmentava ja tukeva tyyli. Karisma liitetään persoonallisiin auktoriteettisiin tyyleihin. Johtamistyyleistä voidaan puhua vain päätöksentekoon liittyvillä käsitteillä koska valmentava ja tukeva tyyli lukeutuvat koordinoijan osaksi ja karsmaattinen johtaja voi kuulua mihin tahansa päätöksenteon tyyleistä~\cite{Dhomne:2012:ITL:2382887.2382899}. 

Koordinoivan johtajan ryhmässä päätökset syntyvät ryhmnäjäsenten päätöksistä. Johtaja pitää kuitenkin itsellään päätäntä oikeuden ongelmatilanteissa. Johtaja toimii valmentavana henkilönä, joka pyrkii kehittämään ryhmänsä jäseniä ja samalla kannustamaan heitä päätöksentekoon. Johtajan samaistuminen ryhmäläisen asemaan kuuluu koordinoivan tukevan johtajan tyyliin. Koordinoiva johtamistyyli voi olla aikaa tuhlaavaa, joten se ei sovellu nopeita päätöksiä vaativiin tehtäviin~\cite{Dhomne:2012:ITL:2382887.2382899}. Koordinoiva johtamistyyli ja etenkin tukeva tyyli sopivat ketterien menetelmien ideologiaan~\cite{fowler2001agile}. Avoimessa ohjelmistokehityksessä päätökset tehdään usein suuremman yhteisön mielipiteen perusteella ja hyvässä yhteisymmärryksessä~\cite{1385637}, joten johtajalle on hyötyä koordinoivasta johtamistyylisä.

Autoritaarinen johtaja pitää päätökset omissa käsissään. Johtaja voi kysyä ryhmänjäseniltä mielipidettä kysymyksistä mutta päätökset johtaja tekee itse. Jos ohjelmistotuotantoprojektin johtajalle on annettu ryhmän, jonka tiedollinen ja taidollinen taso on paljon heikompi kuin johtajan oma tiedon ja taidon taso, on autoritaarinen johtaminen tehokasta~\cite{Dhomne:2012:ITL:2382887.2382899}. Ketterisä mentelmissä päätöksenteon vastuu on annettu kehittäjäryhmille, joten autoritaarinen johtamistapa rikkoo ketterän manifestin periaatteita~\cite{fowler2001agile}.

Laissez-faire on johtamistyyleistä projektin kannalta usein heikoin. Johtaja ei osallistu päätöksen tekoon. Johtaja jättää oman panostuksensa projektiin minimiin. Johtajat, jotka vastaavat useasta projektista saman aikaisesti ajautuvat, joissain projekteissaan tähän tyyliin oman kiinnostuksen ja ajan puutteen takia~\cite{Dhomne:2012:ITL:2382887.2382899}. Laissez-faire johtaminen voi toimia ketterässä kehityksessä koska ketterien menetelmien perustana on itsestään organisoituva kehitysryhmät~\cite{fowler2001agile}. Avoimessa kehityksessä aktiivinen johtaminen on tuloksellisuudne kannalta oleellista, joten Laissez-faire johtaminen ei ole avoimessa ohjelmistokehityksessä tehokasta~\cite{Li:2006:MOS:1125170.1125182}.


\subsection{Ylemmän hallinon vaikutus}

Ohjelmsitotuotanprojektiryhmä on osa suurempaa organisaatiota, jonka alaisena projekti toimii. Organisaation ylemmällä hallinnolla on vaikutusta ohjelmistotuotantoprojekti johtamiseen ja ohjemistotuotantoprojektin tuloksellisuuteen. Mitä ylämmän organisaation tulee huomioda johtamisessaan, jotta ohjelmistotuotantoprojekti onnistuu? Organisaation ylemmällä hallinolla tarkoitetaan organisaation hallitusta ja toimitusjohtajaa, joiden vastuulla on koko organisaation strateginen suunta~\cite{McLeod:2011:FAS:1978802.1978803}.

Ylempi hallininto vaikuttaa ohjelmsituotanprojektin budjettiin ja aikatauluun~\cite{McLeod:2011:FAS:1978802.1978803}. Kiirreellinen aikataulu ja tiukka budjetti ovat syitä ohjelmistotuontaprojektin epäonnistumiselle. Seamanin ja Goun tutkimuksessa ~\cite{Guo:2008:SSP:1414004.1414046} projektin johtajat ilmoittivat aikataulun ja budjetin kolmen tärkeimmän syyn joukkoon prosessin muutokeslle kesken projektin. 

Organisaation ylempi hallinto tarjoaa projektin johtajalle käyttöön tietyt henkilöstö resurssit. Projektin johtaja joutuu valitsemaan kehitysryhmänsä ylemmän organisaation tarjoamista kehittäjistä. Projektin kehitysryhmästä tulee harvoin optimaalinen henkilöstön osalta~\cite{Dhomne:2012:ITL:2382887.2382899}. Asiaakkaan osallistumiseen ja aktiivisuuteen projektissa on ylemmällä hallinnolla vaikutusta~\cite{McLeod:2011:FAS:1978802.1978803}.

Organisaatiolla on usein vakiintuneet kehityst menetelmät, joita organisaatiossa käytetään~\cite{McLeod:2011:FAS:1978802.1978803}. Asennoituminen työtekijöihin ja työntekijöiden asennoitumiseen työhön vaikuttaa organisaation yleinen ilmapiiri ja samalla heikko johtaminen ylemmältä tasolta vaikuttaa myös kehitysryhmään ja kehitysryhmän johtamiseen.  

Organisaation ylempi johto voi asettaa rangaistus menetelmiä joiden avulla hallita tuotettavien ohjelmistojen laatua. Yleinen käsitys rangaistuksista on että ne laskevat kehittäjien motivaatiota. Wang ja Zhang kirjoittavat tutkimuksessaan ~\cite{Wang:2010:PPP:1810295.1810302} että rangaistus politiikka voi tehostaa ohjelmisotuotantoprojektin toimintaa ja tuotteen laatua. Wang ja Zhang antavat johtajalle kolme ohjetta joiden avulla rangaistuksista saadaan toimivia. Rangaistus politiikan tulee olla oikeudenmukaista ja hyvin suunniteltua. Ohjeistus on tärkeää. Kehittäjät tulee saada uskomaan että menetelmän tarkoituksena ei ole säästää rahaa pienentämällä palkkoja vaan saada kehittäjistä motivoituneempia. Kolmas ja tutkijoiden mielestä hyvin oleellinen sääntö on, että rangaistus ei saa ylittää tiettyä osuutta kehittäjien palkasta. Esimerkiksi maksimi rangaistus 5% kuukausipalkasta.    



\subsection{Kehitysryhmän johtaminen}

Ohjelmistotuotanprojektin johtaminen kehitysryhmän tasolla jakautuu useaan vaiheeseen. Projekti alkaa ryhmän kokoamisvaiheella. Kehitysvaiheessa ryhmän kohdatessa riskejä on johtajan vastuulla löytää ratkaisut riskeihin. Debnath ja kumppanit muuttavat artikkelissaan ~\cite{4017705} riskien hallinan muutoksen hallinnaksi ja johtamiseksi.  

Ohjelmistotuotantoprojekteissa johtajan tehtävänä on motivoida muita. Johtajan tehtäviin voidaan lukea mm. organisointi, innovointi, arvionti, viiemistelijä  ja yleisellä tasolla tuotoksesta vastaaminen~\cite{4017705}. Johtamisen tehtävään voidaan liittää myös resurssien hallinan ja projektin aikataulutuksen ~\cite{Dhomne:2012:ITL:2382887.2382899}. Projektin johtajan yhtenä tehtävänä voidaan pitää välikätenä toimimiminen ryhmän ja ulkoisten osallistujien välillä~\cite{McLeod:2011:FAS:1978802.1978803}.  Projektin johtajan ammattitaidolla ja projektin johtamisen työkaluilla on suuri vaikutus ohjelmistotuotanprojektin tuloksellisuuteen~\cite{McLeod:2011:FAS:1978802.1978803}. 

Tehokas ja toimiva ryhmä on olennainen osa onnistunutta projektia. Projektin johtajalla on mahdollisuus vaikuttaa ryhmän muodostamiseen ja mahdollisuus kehittää ryhmän koheesiota. Parempi ryhmä koheesio vaikuttaa positiivisesti ryhmän tuotoksellisuuteen~\cite{bahli2005group}~\cite{McLeod:2011:FAS:1978802.1978803}. Ryhmän toimintaan vaikuttavia tekijöitä, jotka tekevät ryhmästä toimivan ovat yhteinen tavoite, avoin kommunikaatio, rakentavat konfliktien ratkaisut, yleinen luottamus, uskominen synergiaan, avustaminen, kunnioitus ja hyvä johtaja~\cite{4017705}.

Projektin johtajan tulee huomioda projekti ryhmän koko muodostaessaan ryhmnää. Projekti ryhmän koko voi määräytyä ylemmän organisaation puolelta~\cite{McLeod:2011:FAS:1978802.1978803}. Projekti ryhmä joka koostuu suuresta kehittäjä joukosta koetaan riski tekijäksi projektin kannalta~\cite{McLeod:2011:FAS:1978802.1978803}. Projekti ryhmän koon lisäksi johtajan tulee huomioda ryhmäläisten kokemus keskenäisestä toiminnasta. Ihmiset, jotka eivät ole toimineet ennen yhdessä koetaan ryhmän kannalta riksi tekiijöiksi~\cite{McLeod:2011:FAS:1978802.1978803}.



\section{Ryhmän kokoaminen}


Ohjelmistotuotantoprojektin johtamisessa on suuressa roolissa projektin alussa tapahtuva ryhmän kokoaminen. Ryhmän kokoamissen vaikuttaa monet seikat joita johtajan tulisi ottaa huomioon kootessaan ryhmäänsä. Ryhmän kokoamisella ja perusteilla, joita ryhmän kokoajan käyttää on suuri merkitys projektin onnisumiselle~\cite{daSilva2012}. 

Da Silva ja kumppanit tutkivat projektiryhmän kokoajien syitä henkilöiden valinnalle ja niiden vaikutusta ohjelmistotuotantoprojektin onnistumiseen~\cite{daSilva2012}~\cite{francca2009quantitative}. Ohjelmistotuotantoprojekteissa ryhmän kokoamisesta vastaa usein projektipäälikkö tai henkilöstö hallinosta vastaava henkilö. Käytän projektin kokoajasta nimitystä projektipäällikkö. Tutkimuksensa ensimmäisessä vaiheessa da Silva ja kumppanit tutkivat kahdeksaa kriteeriä joiden perusteella projektipäälliköt valitsevat henkilöitä projekteihinsa~\cite{francca2009quantitative}. Myöhemmässä tutkimuksessa he kasvattivat kriteereiden määrä kymmeneen~\cite{daSilva2012}. Kymmenen kriteeriä joiden vaikutusta henkilöstön valinnassa projektipäälliköt käyttivät ovat persoona, käyttäytyminen, tekninen profiili, asiakaan tärkeys, tuotoksellisuus, käytettävyys, yksilön hinta projektille, projektin tärkeys, suositukset ja tehtävään sopivuus. Tarkemmat kuvaukset kriteereistä voi lukea da Silvan ja kumppaneiden tutkimuksesta~\cite{daSilva2012}.

Tutkimuksessaan da Silva ja kumppanit huomasivat että johtajat painottivat valinta kriteereissään teknistä profiilia ja  projektin tärkeyttä. Tutkimuksessa kuitenkin huomattiin että suurin vaikutus projektin onnistumiseen oli persoonallisuudella ja käyttäytymisellä~\cite{daSilva2012}. Projekti päälliköt käyttävät usein entuudestaan tuttuja tapoja henkilöiden valitsemiseen. Osaamisen ja tiedon puute vaikuttavat inhimillisten tekijöiden käyttämiseen valinta prosessissa. Ohjelmistoyritysten tulisi huomioda inhimillisten tekijöoden vaikutus ja kouluttaa johtajiaan huomioimaan henkilöresurssit~\cite{daSilva2012}. 

Löytääkseen oikeat henkilöt oikeaan paikkaan voi ohjelmistotuotantoprojektien johtaja käyttää apunaan "I Opt" menetelmää ~\cite{Dhomne:2012:ITL:2382887.2382899}. I Opt-menetelmä on esitelty tarkemmin artikkelissa~\cite{ kliem1996teambuilding}. Menetelmässä ryhmän jäsenet arviodaan kyselyn perusteella. Jokainen ryhmäläinen asetetaan luokkaan tyylinsä mukaan. Nämä neljä luokkaa ovat toimintaan suuntautunut, looginen prosessoija, analysoija ja innovoija~\cite{ kliem1996teambuilding}. Johtajan ja ryhmän valinta olisi optimaalista jos jokaiseen tehtävään voitaisiin valita tehtävätyyppiin sopivimmat henkilöt~\cite{Dhomne:2012:ITL:2382887.2382899}.

Ketterissä menetelmissä ryhmän kokoamisen merkitys ei oli yhtä suuri verrattuna perinteisiin menetelmiin esimerkiksi vesiputousmalliin~\cite{daSilva2012}. Da Silvan ja kumppaneiden mielestä tämä voi johtua ketterien menetelmien joustavuudesta. Ketterillä ryhmillä on usein kyky muuttua ja samalla ryhmästä muovautuu tehokkaampi. Toisaalta da Silva ja kumppanit mainitsevat että jos edellä mainittu olettamus ketteristä menetelmistä pitää paikkansa tulee ryhmän jäsenillä olla inhimillisiä kykyjä~\cite{daSilva2012}. 


\section{Riskien hallinta vs muutos johtaminen}


\section{Johtajan persoonallisuus}

Ohjemistotuotantoprojekteissa johtajalla on monia mahdollisuuksia vahvistaa käytettävissä olevan ryhmänsä toimintaa. Ryhmä suoritukseen vaikkuttaa johtajan persoona. Johtajan persoonalle hyödyllisiä ominaisuuksia ovat kärsivällisyys, joustavuus, taktikointi, kommunikointitiadot, huumorintaju, auktoriteetti ja tieto. Monet näistä piirteistä vaikuttavat toisiinsa. Tehokaalla projektin johtajalla on kokemusta ja tietämystä käytettävistä tekniikoista ja ihmissuhdetaidoista ~\cite{McLeod:2011:FAS:1978802.1978803}. Johtajan ei tarvitse kuitenkaan olla ryhmästä se henkilö, jolla on paras tuntemus käsiteltävästä asiasta~\cite{4017705}. Johtajan persoonnallisuuden vaikutuksista projektiin on tutkittu vähäisestii~\cite{Wang:2009:PMP:1639950.1640049}. 

\subsection{Five Factor personality model ja Mohan Thite's model }

Ihmisen tavat, uskomukset ja asenteet ovat osa ihmisen persoonallisuutta. Ihmisen persoonallisuutta kuvaavia atribuutteja on useita eikä niiden avulla ole helppoa saada todellista kuvaa ihmisen persoonallisuudesta. Eräs tapa tutkia ihmisen persoonaa ja sen vaikutusta on Five Factor Model (FFM)~\cite{barrick2006big}. Tämän menetelmän avulla voidaan kuvata myös johtajaan persoonallisuutta. FFM:ssä jokainen persoonnallisuuden piirre sisällytetään johonkin viidestä laajemmasta tekijästä. Viisi laajempaa tekijää, joita mallissa käytetään ovat avoimmuus, neuroottisuus, miellyttävyys, tunnollisuus ja sosiaalisuus ~\cite{barrick2006big}.

Wang ja Li yhdistivät tutkimuksessaan FFM:nä Mohan Thiten kehittämään malliin jonka avulla voidaan tutkia projektin johtamisen ja projektin onnistumisen välistä sudetta ~\cite{Wang:2009:PMP:1639950.1640049}. Luomansa mallin avulla Wang ja Li tutkivat johtajan persoonallisuuden vaikutusta ohjelmistotuotantoprojektin onnistumiseen. 

Johtajan avoimmuuden on vaikutusta projektin onnistumiseen. Avoimuudella FFM:ssa tarkoitetaan kykyä löytää epätavallisia ideoita, mielikuvituksellisuus, uteliaisuus, emootiot ja seikkailullisuus~\cite{Wang:2009:PMP:1639950.1640049}. Avoimet ihmiset soveltuvat muutosjohtamiseen koska tämän kaltaiset ihmiset ovat valmiita yrittämään uusia asioita. Johtaja, joka ei ole avoin voi ohittaa muuttuvat tekniset vaatimukset ja mahdollisuudet. 

Neuroottisuus kuvastaa useita negatiivisia piirteitä. Näitä piirteitä ovat esimerkiksi viha, levottomuus ja masentuneisuus. Wang ja Li huomasivat että neuroottiset luonteenpiirteet ovat negatiivisesti sidottuina projektin menestymiseen~\cite{Wang:2009:PMP:1639950.1640049}. Neuroottiset henkilöt suhtautuvat usein negatiivisesti työtään kohtaan. Neuroottinen johtaja on näissä tilanteissa usein johtamistyyliltään Laissez-faire tyylinen johtaja.

Johtajan miellyttävyyteen liitetään piirteitä, jotka rakentava ryhmän jäsenten luottamusta johtajaan. Piirteitä ovat esimerkiksi kiltteys, auttavainen, yhteistyökykyinen ja huomaavaisuus~\cite{Wang:2009:PMP:1639950.1640049}. Miellyttävä projektin johtaja rakentaa ryhmän koheesiota. Ryhmä jonka sisäinen koheesio on parempi saavuttaa projektin tavoitteet todennäköisemmin~\cite{bahli2005group}. Wang ja Li saivat samansuuntaisen päätelmän että miellyttävällä johtajalla on positiivinen vaikutus projektin onnistumiseen.

Itsekuri, velvollisuudentuntoisuus ja tavoitteellisuus johtajan piirteinä vaikuttavat projektin tulokseen positiivisesti~\cite{Wang:2009:PMP:1639950.1640049}. Nämä luenteen piirteet kuuluvat FFM:ssä tunnolisuuden tekijän alle. Johtaja toimii esimies asemassa ja samalla alaisilleen esimerkkinä, joka mahdolllisesti korottaa kehittäjäryhmän tuloksellisuutta. 

Sosiaalinen johtaja toimii läheisessä yhteydessä kehittäjäryhmäänsä ja pyrkii luomaan lämpimän suhteen alaisiinsa. Ohjelmistotuotantoprojekteissa johtaja toimii asikkaan ja muiden yhteisyö tahojen kanssa yhteistyössä~\cite{McLeod:2011:FAS:1978802.1978803}, joten sosiaalisella johtajalla on näissä tilanteissa paremmat edellytykset onnistua. Tutkimuksessaan Wang ja Li saivat selville että johtajan sosiaalisuus ja energisyys vaikuttavat positiivisesti  projektin tuloksiini~\cite{Wang:2009:PMP:1639950.1640049}. 

\subsection{Johtamistyylit}

Jokainen projektin johtaja on henkilönä erilainen ja johtajilla on erilaiset johtamistyylit. Johtamistyylejä vaikutusta projektiryhmän rakentamiseen ja projektin tuloksellisuuteen on tutkittu. 

Artikkelissaan ~\cite{Dhomne:2012:ITL:2382887.2382899} Dhomme ja Hall jakavat johtamistyylit päätöksenteko-, aktiivinen johtaminen- ja persoonalliset auktoriteettiset tyylit. Päätöksenteko tyylejä ovat koordinoija, autoritaarinen ja laissez-faire. Aktiivisiin tyyleihin kirjataan valmentava ja tukeva tyyli. Karisma liitetään persoonallisiin auktoriteettisiin tyyleihin. Johtamistyyleistä voidaan puhua vain päätöksentekoon liittyvillä käsitteillä koska valmentava ja tukeva tyyli lukeutuvat koordinoijan osaksi ja karsmaattinen johtaja voi kuulua mihin tahansa päätöksenteon tyyleistä~\cite{Dhomne:2012:ITL:2382887.2382899}. 

Koordinoivan johtajan ryhmässä päätökset syntyvät ryhmnäjäsenten päätöksistä. Johtaja pitää kuitenkin itsellään päätäntä oikeuden ongelmatilanteissa. Johtaja toimii valmentavana henkilönä, joka pyrkii kehittämään ryhmänsä jäseniä ja samalla kannustamaan heitä päätöksentekoon. Johtajan samaistuminen ryhmäläisen asemaan kuuluu koordinoivan tukevan johtajan tyyliin. Koordinoiva johtamistyyli voi olla aikaa tuhlaavaa, joten se ei sovellu nopeita päätöksiä vaativiin tehtäviin~\cite{Dhomne:2012:ITL:2382887.2382899}. Koordinoiva johtamistyyli ja etenkin tukeva tyyli sopivat ketterien menetelmien ideologiaan~\cite{fowler2001agile}. Avoimessa ohjelmistokehityksessä päätökset tehdään usein suuremman yhteisön mielipiteen perusteella ja hyvässä yhteisymmärryksessä~\cite{1385637}, joten johtajalle on hyötyä koordinoivasta johtamistyylisä.

Autoritaarinen johtaja pitää päätökset omissa käsissään. Johtaja voi kysyä ryhmänjäseniltä mielipidettä kysymyksistä mutta päätökset johtaja tekee itse. Jos ohjelmistotuotantoprojektin johtajalle on annettu ryhmän, jonka tiedollinen ja taidollinen taso on paljon heikompi kuin johtajan oma tiedon ja taidon taso, on autoritaarinen johtaminen tehokasta~\cite{Dhomne:2012:ITL:2382887.2382899}. Ketterisä mentelmissä päätöksenteon vastuu on annettu kehittäjäryhmille, joten autoritaarinen johtamistapa rikkoo ketterän manifestin periaatteita~\cite{fowler2001agile}.

Laissez-faire on johtamistyyleistä projektin kannalta usein heikoin. Johtaja ei osallistu päätöksen tekoon. Johtaja jättää oman panostuksensa projektiin minimiin. Johtajat, jotka vastaavat useasta projektista saman aikaisesti ajautuvat, joissain projekteissaan tähän tyyliin oman kiinnostuksen ja ajan puutteen takia~\cite{Dhomne:2012:ITL:2382887.2382899}. Laissez-faire johtaminen voi toimia ketterässä kehityksessä koska ketterien menetelmien perustana on itsestään organisoituva kehitysryhmät~\cite{fowler2001agile}. Avoimessa kehityksessä aktiivinen johtaminen on tuloksellisuudne kannalta oleellista, joten Laissez-faire johtaminen ei ole avoimessa ohjelmistokehityksessä tehokasta~\cite{Li:2006:MOS:1125170.1125182}.



\section{Yhteenveto}







 





Esimerkkilause ja lähdeviite~\cite{Zhang:2011:ECL:2047594.2047666}
Esimerkkilause ja lähdeviite~\cite{Dhomne:2012:ITL:2382887.2382899}
Esimerkkilause ja lähdeviite~\cite{Augustine:2005:APM:1101779.1101781}
Esimerkkilause ja lähdeviite~\cite{Chow2008961}~\cite{Li:2006:MOS:1125170.1125182}
Esimerkkilause ja lähdeviite~\cite{4755768}





% --- Back matter ---
%

% bibtex is used to generate the bibliography. The babplain style
% will generate numeric references (e.g. [1]) appropriate for theoretical
% computer science. If you need alphanumeric references (e.g [Tur90]), use
%
% \bibliographystyle{babalpha}
%
% instead.

\bibliographystyle{babplain}
\bibliography{references-fi}


\end{document}