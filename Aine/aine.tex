% --- Template for thesis / report with tktltiki2 class ---

\documentclass[finnish]{tktltiki2}

% tktltiki2 automatically loads babel, so you can simply
% give the language parameter (e.g. finnish, swedish, english, british) as
% a parameter for the class: \documentclass[finnish]{tktltiki2}.
% The information on title and abstract is generated automatically depending on
% the language, see below if you need to change any of these manually.
% 
% Class options:
% - grading                 -- Print labels for grading information on the front page.
% - disablelastpagecounter  -- Disables the automatic generation of page number information
%                              in the abstract. See also \numberofpagesinformation{} command below.
%
% The class also respects the following options of article class:
%   10pt, 11pt, 12pt, final, draft, oneside, twoside,
%   openright, openany, onecolumn, twocolumn, leqno, fleqn
%
% The default font size is 11pt. The paper size used is A4, other sizes are not supported.
%
% rubber: module pdftex

% --- General packages ---

\usepackage[utf8]{inputenc}
\usepackage{lmodern}
\usepackage{microtype}
\usepackage{amsfonts,amsmath,amssymb,amsthm,booktabs,color,enumitem,graphicx}
\usepackage[pdftex,hidelinks]{hyperref}

% Automatically set the PDF metadata fields
\makeatletter
\AtBeginDocument{\hypersetup{pdftitle = {\@title}, pdfauthor = {\@author}}}
\makeatother

% --- Language-related settings ---
%
% these should be modified according to your language

% babelbib for non-english bibliography using bibtex
\usepackage[fixlanguage]{babelbib}
\selectbiblanguage{finnish}

% add bibliography to the table of contents
\usepackage[nottoc,numbib]{tocbibind}
% tocbibind renames the bibliography, use the following to change it back
\settocbibname{Lähteet}

% --- Theorem environment definitions ---

\newtheorem{lau}{Lause}
\newtheorem{lem}[lau]{Lemma}
\newtheorem{kor}[lau]{Korollaari}

\theoremstyle{definition}
\newtheorem{maar}[lau]{Määritelmä}
\newtheorem{ong}{Ongelma}
\newtheorem{alg}[lau]{Algoritmi}
\newtheorem{esim}[lau]{Esimerkki}

\theoremstyle{remark}
\newtheorem*{huom}{Huomautus}


% --- tktltiki2 options ---
%
% The following commands define the information used to generate title and
% abstract pages. The following entries should be always specified:

\title{Aine: Johtaminen ja johtajuus, rooli ja vaikutus ohjelmistotuotantoprojekteissa: ketterät mentelmät ja avoin kehitys}
\author{Mika Kivi}
\date{\today}
\level{Aine}
\abstract{Tiivistelmä.}

% The following can be used to specify keywords and classification of the paper:

\keywords{johtaminen, ohjelmistotuotanprojektin johtaminen, muutos johtaminen}
\classification{} % classification according to ACM Computing Classification System (http://www.acm.org/about/class/)
                  % This is probably mostly relevant for computer scientists

% If the automatic page number counting is not working as desired in your case,
% uncomment the following to manually set the number of pages displayed in the abstract page:
%
% \numberofpagesinformation{16 sivua + 10 sivua liitteissä}
%
% If you are not a computer scientist, you will want to uncomment the following by hand and specify
% your department, faculty and subject by hand:
%
% \faculty{Matemaattis-luonnontieteellinen}
% \department{Tietojenkäsittelytieteen laitos}
% \subject{Tietojenkäsittelytiede}
%
% If you are not from the University of Helsinki, then you will most likely want to set these also:
%
% \university{Helsingin Yliopisto}
% \universitylong{HELSINGIN YLIOPISTO --- HELSINGFORS UNIVERSITET --- UNIVERSITY OF HELSINKI} % displayed on the top of the abstract page
% \city{Helsinki}
%


\begin{document}

% --- Front matter ---

\maketitle        % title page
\makeabstract     % abstract page

\tableofcontents  % table of contents
\newpage          % clear page after the table of contents


% --- Main matter ---

\section{Johdanto}

Tietokoneet ja tietokoneiden kaltaiset laitteet ovat yleistyneet ihmisten käytössä. Tietokoneita ja laitteita, jotka toimivat ohjelmistoilla löytyy ihmisten kodeista ja työpaikoilta. Näihin laitteisiin kehitetään jatkuvasti uusia ohjelmistoja ja vanhoja ohjelmistoja kehitetään paremmiksi. Ohjelmistojen kehittäminen on kuitenkin monimutkainen prosessi, jonka onnistumiseen vaikuttavat monet tekijät.

Ohjelmistotuotanto on ihmislähtöistä toimintaa, jossa ihmiset tekevät ohjelmistoja ihmisille~\cite{Wang:2010:PPP:1810295.1810302}(kaivettava alkup?). Ihmisillä on siis suuri vaikutus projektien onnistumiseen ja epäonnistumiseen. Muita projektin tulokseen vaikuttavia tekijöitä ovat projektin toimintavat, käytettävä kehitysprosessi ja ympäristö, jossa projektia toteutetaan~\cite{McLeod:2011:FAS:1978802.1978803}. 


Tässä aineessa pyrin selvittämään ihmisten ja toimintatapojen vaikutusta projektin lopputuloksiin. Aine keskittyy johtamisen ja johtajuuden vaikutuksiin ohjelmistotuotantoprojektiin. Tavoitteena on myös löytää keinoja joiden avulla projektin johtajalla on mahdollisuus vaikuttaa projektin lopputulokseen. Ohjelmistotuotantoprojektin johtamista tutkitaan tarkemmin ketterien menetelmien ja avoimen kehityksen näkökulmista.

Johtamista käsitellään eri tasoilla. Käsiteltävänä on kehitysryhmän sisäinen ja ulkoinen johtaminen. Aineessa luodaan myös katsausta ylemmän organisaation johtamisen vaikutuksesta ryhmän toimintaan. 

Aine jakautuu viteen osaan. Ensimmäisessä osassa esitellään käsiteltävät kehitys menetelmät. Toinen osa keskittyy ohjelmistotuotanprojektin johtamiseen yleisellä tasolla. Seuraavat kaksi osiota syvennytään johtamiseen ketterissä menetelmissä sekä avoimessa ohjelmistokehityksessä. Viimeisessä osassa etsitään keinoja kuinka johtaja ja johtajan persoonallisuus voivat vaikuttaa ohjelmistotuotantoprojektin tuloksiin.
  



\section{Ohjelmsitotuotantoprojektin kehitys menetelmät}

Ohjelmistojentuottamiseen on kehittynyt erialisia kehitys menetelmiä. Tässä aineessa käsittellään tarkemmin johtamista ketterien menetelmien ja avoimen ohjelmistokehityksen näkökulmasta. Tässä luvussa esitellään kyseisten menetelmien ideaologioita ja persukäsitteistöä. Ketteristä menetelmstä esitellään yksittäisiä menetelmiä ja avoimesta kehityksestä esitellään tunnetuja projekteja.

\subsection{Ketterät menetelmät}



Ketterät menetelmät (agile methodologies) on kokoelma erilaisia kehitys menetelmiä, joiden ideologia perustuu ketterän manifestin ajatuksiin~\cite{fowler2001agile}. Menetelmien tarkoituksena on pyrkiä mahdollisimman hyvään asiakas tyytyväisyyteen. Tämän tavoitteen saavuttamiseksi ohjelmistosta pyritään tarjoamaan asiakkaalle toimiva versio jatkuvasti kehityksen aikana~\cite{fowler2001agile}. Ketterät menetelmät perustuvat iteratiiviseen kehitykseen, jossa kehitystä tehdään sykäyksissä. Sykäyksen jälkeen tarkastellaan aikaan saannosta ja päätetään mitä seuraavassa sykäyksessä tehdään. 

Manifesti os syntynyt vuonna 2001 17 tunnetun ohjelmistoalanammattilaisen tapaamisessa. Ryhmän jäsenten oli tarkoitus löytää yheteinen käsitys uusista menetelmistä, joita oli ohjelmistojen kehitykseen syntynyt. Ketteriksi menetelmiksi voidaan lukea muunmuassa Extreme programing(XP), SCRUM, Crystal  ja Feature-Driven development~\cite{fowler2001agile}. Tässä aineessa ketteristä menetelmistä käsitellään tarkemmin johtamista XP ja SCRUM kehyksissä. 

Ketteriä menetelmien toivottiin ratkaisevan ohjelmistotuotantoprojektien ongelmat. Manifestin kirjoittajat kuitenkin korostava että he eivät ole löytäneet täydellistä ratkaisua ohjelmistojen kehityksen ongelmiin. Ketteriä menetelmiä on arvosteltu aikatualun ja budjetin ylittämisestä~\cite{Guo:2008:SSP:1414004.1414046}.   

Ketterät menetelmät korostavat ihmisten välistä vuorovaikutusta. Vuorovaikutuksen korostaminen vaikuttaa oleellisesti johtamiseen. Manifesti kehoittaa kasvokkain tapahtuvaan vuorovaikutukseen. Vuorovaikutusta asiakkaan ja kehitysryhmän välillä korostetaan.

 SCRUM menetelmän periaateet esiteltiin ensimmäisen kerran 1986. Periaateet esitteli Hirotaka Takeuchi ja Ikujiro Nonaka kirjassaan "The New New Product Development Game"~\cite{nonaka1986new}. SCRUM perustuu kehitysryhmän itsehalllinnalle. Kehitys ryhmällä ei ole varsinaista johtaa. Päätökset ryhmä tekee yhdessä. SCRUM:ssa tärkeitä rooleja ovat SCRUM master, SCRUM team ja product owner ~\cite{schwaber2002agile}. 

Extreme Programing (XP) on kehittänyt Kenet Beck joka artikkelissaan "Embracing change with extreme programming" menetelmän ideologiaa~\cite{796139}. XP:ssä säännökset toiminnalle ovat tarkemmat kuin SCRUM:ssa. Projektin työskentely tila on XP:ssa avoin, jossa keskellä on tietokoneet, joilla ohjelmointi toteutetaan. Ohjelmointi toteutetaan pari ohjlemointina. Jokaisella parilla on vain yksi näyttö, näppäimistö ja hiiri. Ohjelmen osien kehitystö ei ole osoitettu tietyille henkilöille vaan jokaisella ryhmänjäsenellä on oikeus muuttaa ohjelman kaikkia osa-alueita. XP:ssä testauksen merkitystä korostetaan. Asiakas osallistuu projektiin täyspäiväisesti eli on osa kehitys ryhmään~\cite{796139}.  



\subsection{Avoin ohjelmistokehitys}

Avoimet ohjelmistot ovat ohjelmia joita kehittävät ryhmät, jotka koostuvat vapaaehtoisesista ihmisistä joille ei makseta palkkaa ohjelmiston kehittämisestä. Ryhmien jäsenistö on maantieteellisesti hajaantunutta. Kommunikaatio ja osallistuminen kehitystoimintaan tapahtuu internetin välityksellä.

Avoin ohjelmisto kehitys koetaan tärkeäksi, josta kertoo myös monien avointen projektien rooli tämän päivän maailmassa~\cite{Li:2006:MOS:1125170.1125182}. Tunnettuja avoimen ohjelmistokehityksen tuotteita ovat Linux, Apache, Mozilla Firefox, Netbeans ja Eclipse. 

Avoimille ohjelmistoille löytyy usein kaupallinen vastine, jota vastaan avoinohjelmisto kilpailee~\cite{Luther:2008:LOC:1460563.1460619}(Alkuperäinen tarvitaan). Linux, joka on avoin käyttöjärjestelmä on vastine Microsoftin Windows käyttöjärjestelmälle ja Netbeans sekä Eclipse ovat Microsoftin Visual studion avoimia vastikkeita. Avoimesti kehitetyt ohjelmistot ovat pääosin ilmaisia joten niillä on hinnan perusteella etulyönti asema kaupallisiin kilpailijoihinsa nähden. 

Avoimen ohjelmitokehitysprojektin johtamista tarkastellaan tässä aineessa kolmen eri projektin näkökulmasta. Käsiteltäviä projekteja ovat Linux, Apache ja Netbeans. Käsiteltävien projektien ohella luodaan katsasus verkosssa toimivien kehitys yhteisöiden johtamisesta. 

Linux... on kehittynyt (kun artsu luettu :P)

Apache... On kehittynyt (kun artsu luettu :P)

Netbeans on kehitysympäristö jota kehittää ja ylläpitää Netbeans.org-yhteisö. Netbeanssin kehitys on aloitettu vuonna 1996. Netbeans oli alunperin opiskelija lähtöinen projekti,  jonka  Sun julkaisi avoimeen kehitykseen~\cite{1385637}. Kehittäjät osallistuvat yhteisön toimintaan sähköpostilistoilla, korjaamalla ohjelmistoa, päivittämällä nettsivuja tai tekemällä käännöstä ohjelmsta. 

\section{Johtaminen ja johtajuus ohjelmistotuotannossa}

Ohjelmistotuotannossa johtamisella voidaan tarkoittaa useita erilaisia johtamisen tasoja. Tässä osassa selvitetään kuinka erilaiset johtamisen tasot vaikuttavat ohjelmistotuotantoprojektiin. Aluksi puhutaan johtamisesta yleisesti ja siitä missä johtamista esiintyy. Tämän jälkeen käsitellään ylemmän hallinon johtamisen vaikutusta ohejlmistotuotanprojektin toimintaan. Tämän jälkeen siirrytään itse projektin johtamiseen. Tämän luvun lopussa käydään läpi johtajan persoonallisuuden vaikutusta johtamiseen.

\subsection{Yleinen johtaminen}

Tässä tullaan avamaan johtamisen käsitettä yleisesti?

\subsection{Ylemmän hallinon vaikutus}

Kuinka ylempi organisaatio vaikuttaa projektin johtamiseen ja johtajaan?

\subsection{Kehitysryhmän johtaminen}

Ohjelmistotuotantoprojekteissa johtajan tehtävänä on motivoida muita. Johtajan tehtäviin voidaan lukea mm. organisointi, innovointi, arvionti, viiemistelijä  ja yleisellä tasolla tuotoksesta vastaaminen. ~\cite{4017705}

\subsection{Johtajan persoonallisuus}

Millainen johtajan persoonnalisuuden tulisi olla?

Ohjemistotuotantoprojekteissa johtajalla on monia mahdollisuuksia vahvistaa käytettävissä olevan ryhmänsä toimintaa. Ryhmä suoritukseen vaikkuttaa johtajan persoona. Johtajan persoonalle hyödyllisiä ominaisuuksia ovat kärsivällisyys, joustavuus, taktikointi, kommunikointitiadot, huumorintaju, auktoriteetti ja tieto. Monet näistä piirteistä vaikuttavat toisiinsa. Johtajan on helmpompaa saavuttaa auktoriteetin asema ryhmssään, jos hänellä on tietämystä aihealueesta. Johtajan ei tarvitse kuitenkaan olla ryhmästä se henkilö, jolla on paras tuntemus käsiteltävästä asiasta.~\cite{4017705} 

\section{Johtajan rooli ketterissä menetelmissä}

Ketterissä menetelmissä projektin johtajan rooli eroo suurelta osin yleisestä johtamisen käsitteestä. Tässä luvussa käsitellään projektin johtamista ketterissä menetelmissä. Johtajan roolia tarkastellaan tarkemmin kahdessa ketterässä menetelmässä. 

\subsection{Scrum}
Esimerkkilause ja lähdeviite~\cite{4755768}

\subsection{Extreme programing(XP)}

\section{Jothamisen tehtävät avoimessa ohjelmisto kehityksessä}

Avoimessa ohjelmisto kehityksessä johtajan rooli on monimutkainen. Tässä luvussa selvitetään johtajan tehtävää avoimessa ohjelmisto kehityksessä. Pyrin myös löytämään ratkaisu keinoja joilla avoimen ohjelmisto prosessin johtaja voi tehostaa prosessin toimintaa.

\subsection{Johtamis käytänteet}

Avoimessa ohjelmistokehityksessä johtajaa ei usein määräydä ylemmän organisaation määräyksestä. Johtavan henkilön asemaan voidaan päästä kahdella tavalla. Kun edellinen projektin johtaja päättää jättää projektin kehittämisen hän voi määrätä tai ehdottaa itselleen jatkaa. Toisessa tilanteessa yhteisö, joka ohjemistoa kehittää valitsee jäsenistöstään iteselleen johtavan henkilön. ~\cite{Li:2006:MOS:1125170.1125182}

\subsection{Johtajan vaikutus prosessiin}

Johtajan tärkeimmäksi tehtäväksi avoimessa ohjelmistokehityksessä  voidaan sanoa yhteisön motvoinnin. Kuinka saada ihmiset kontribuoimaan.~\cite{Li:2006:MOS:1125170.1125182}


\section{Johtajan vaikutus ohjelmistotuotantoprojekteissa}

Johtaja mahdollisuudet vaikuttaa ohjelmistotuotantoprosessin lopputulokssen on monet. Tässä luvussa käsitellään tapoja joilla johtaja voi vaikuttaa ohjelmisotuotanprojektin lopputuloksiin. Luvussa käsitellään myös aiheita, joita projektin johtajan tulee ottaa huomioon kehityksen aikan, jotta projekti voi onnistua. 

Johtajan persoonan vaikutusta voidaan tutkia Five Factor personal mallin avulla ja Mohan Thiten mallin avulla~\cite{Wang:2009:PMP:1639950.1640049}(Tähänkin joutuu kaivaa alkuperäisen). 


\section{Yhteenveto}







 




Esimerkkilause ja lähdeviite~\cite{Guo:2008:SSP:1414004.1414046}.
Esimerkkilause ja lähdeviite~\cite{Zhang:2011:ECL:2047594.2047666}
Esimerkkilause ja lähdeviite~\cite{Dhomne:2012:ITL:2382887.2382899}
Esimerkkilause ja lähdeviite~\cite{bahli2005group}
Esimerkkilause ja lähdeviite~\cite{Augustine:2005:APM:1101779.1101781}
Esimerkkilause ja lähdeviite~\cite{Chow2008961}





% --- Back matter ---
%

% bibtex is used to generate the bibliography. The babplain style
% will generate numeric references (e.g. [1]) appropriate for theoretical
% computer science. If you need alphanumeric references (e.g [Tur90]), use
%
% \bibliographystyle{babalpha}
%
% instead.

\bibliographystyle{babplain}
\bibliography{references-fi}


\end{document}