% --- Template for thesis / report with tktltiki2 class ---

\documentclass[finnish]{tktltiki2}

% tktltiki2 automatically loads babel, so you can simply
% give the language parameter (e.g. finnish, swedish, english, british) as
% a parameter for the class: \documentclass[finnish]{tktltiki2}.
% The information on title and abstract is generated automatically depending on
% the language, see below if you need to change any of these manually.
% 
% Class options:
% - grading                 -- Print labels for grading information on the front page.
% - disablelastpagecounter  -- Disables the automatic generation of page number information
%                              in the abstract. See also \numberofpagesinformation{} command below.
%
% The class also respects the following options of article class:
%   10pt, 11pt, 12pt, final, draft, oneside, twoside,
%   openright, openany, onecolumn, twocolumn, leqno, fleqn
%
% The default font size is 11pt. The paper size used is A4, other sizes are not supported.
%
% rubber: module pdftex

% --- General packages ---

\usepackage[utf8]{inputenc}
\usepackage{lmodern}
\usepackage{microtype}
\usepackage{amsfonts,amsmath,amssymb,amsthm,booktabs,color,enumitem,graphicx}
\usepackage[pdftex,hidelinks]{hyperref}

% Automatically set the PDF metadata fields
\makeatletter
\AtBeginDocument{\hypersetup{pdftitle = {\@title}, pdfauthor = {\@author}}}
\makeatother

% --- Language-related settings ---
%
% these should be modified according to your language

% babelbib for non-english bibliography using bibtex
\usepackage[fixlanguage]{babelbib}
\selectbiblanguage{finnish}

% add bibliography to the table of contents
\usepackage[nottoc,numbib]{tocbibind}
% tocbibind renames the bibliography, use the following to change it back
\settocbibname{Lähteet}

% --- Theorem environment definitions ---

\newtheorem{lau}{Lause}
\newtheorem{lem}[lau]{Lemma}
\newtheorem{kor}[lau]{Korollaari}

\theoremstyle{definition}
\newtheorem{maar}[lau]{Määritelmä}
\newtheorem{ong}{Ongelma}
\newtheorem{alg}[lau]{Algoritmi}
\newtheorem{esim}[lau]{Esimerkki}

\theoremstyle{remark}
\newtheorem*{huom}{Huomautus}


% --- tktltiki2 options ---
%
% The following commands define the information used to generate title and
% abstract pages. The following entries should be always specified:

\title{Referaatti artikkelista: Software Projects Leadership: Elements to Redefine "riskmanagement" Scopeand Meaning}
\author{Mika Kivi}
\date{\today}
\level{Referaatti}


% The following can be used to specify keywords and classification of the paper:

%\keywords{avainsana 1, avainsana 2, avainsana 3}
%\classification{} % classification according to ACM Computing Classification System (http://www.acm.org/about/class/)
                  % This is probably mostly relevant for computer scientists

% If the automatic page number counting is not working as desired in your case,
% uncomment the following to manually set the number of pages displayed in the abstract page:
%
% \numberofpagesinformation{16 sivua + 10 sivua liitteissä}
%
% If you are not a computer scientist, you will want to uncomment the following by hand and specify
% your department, faculty and subject by hand:
%
% \faculty{Matemaattis-luonnontieteellinen}
% \department{Tietojenkäsittelytieteen laitos}
% \subject{Tietojenkäsittelytiede}
%
% If you are not from the University of Helsinki, then you will most likely want to set these also:
%
% \university{Helsingin Yliopisto}
% \universitylong{HELSINGIN YLIOPISTO --- HELSINGFORS UNIVERSITET --- UNIVERSITY OF HELSINKI} % displayed on the top of the abstract page
% \city{Helsinki}
%


\begin{document}

% --- Front matter ---

\maketitle        % title page
%\makeabstract     % abstract page

%\tableofcontents  % table of contents
%\newpage          % clear page after the table of contents


% --- Main matter ---

%\section{...}

% Write some science here.

Ohjelmistotuotanto projekteissa on onnistumisen kannalta tärkeässä roolissa on riskien hallinta. Artikkelissa Software Projects Leadership: Elements to Redefine "riskmanagement" Scope and Meaning Depnath ja kumppanit esittelevät oman mallinsa riskien hallintaan ohjelmistotuotanto projekteissa. Heidän näkemyksensä mallin toimivuudesta pohjaa tieteelliseen tutkimukseen sekä tosielämän kokemuksiin ohjelmistotuotanto projekteista. Kirjoittajien mielestä useimmat kirjoissa ja tutkimuksissa esitetyt lähestymistavat ohjelmistojen riskien hallintaa kohtaan eivät sovellu todellisiin projekteihin. Useimmat ongelmat ohjelmistotuotanto projekteissa johtuvat projektin ympäristön muutoksista. 

Tutkimusten mukaan ohjelmistotuotanto projekteissa voidaan riskit jakaa kahteen ryhmään yleiset ongelmat ja projekti spesifit ongelmat. Yleiset ongelmat esiintyvät mahdollisesti useissa eri projekteissa kun taas projekti spesifit ongelmat ovat jokaiselle projektille erilaiset. Kirjoittajien mielestä hyvä tapa kohdata molemman tyyppisiä ongelmia on projektin johtajan ja projektiryhmän ammattitaito.  
 
Kirjoittajat muuttavat artikkelissaan riskien hallinta ongelman muutoksen hallinnan ja johtajuuden ongelmaksi. Muutosten hallinta muuttuu riskien hallinnaksi kun muutosten hallinta taidot ovat puutteelliset. Tämä johtuu usein johtajan johtamistaitojen puutteesta. Useimmat ongelmat ohjelmistotuotanto projekteissa johtuvat projektin ympäristön muutoksista.  Projektin johtajan hidas reagointi muuttuneisiin olosuhteisiin aiheuttaa ongelmien kierteen, joka on vaikea katkaista.
Akateemisessa yhteydessä riskienhallinnan prosessiin liitetään usein neljä vaihetta: riskien tunnistaminen, riskien analysointi, riskien välttämisen suunniteleminen ja riskien tarkkailu. Ohjelmistotuotanto projekteissa riskit jakautuvat useaan eri dimensioon, joihin vaikuttava esimerkiksi ihmiset ja projektissa käytettävät työkalut. Akateeminen tutkimus ei kuitenkaan ota huomioon muutosten hallintaa sen todellisessa mittasuhteessa. Kirjoittajat ehdottavatkin käytettäväksi Kotterin mallia.
 
Kotterin malli perustuu kolmeen muutoksen vaiheeseen. Ensimmäisessä vaiheessa johtajan tulee löytää syy muutokselle ja saada muut uskomaan että tilanne muutkosen jälkeen on parempi kuin ennen muutosta. Johtajien tulee yhdessä muodostaa liittouma, jonka avulla he voivat ohjeistaa muutosta oikeaan suuntaan. Ilman yhtenäistä linjaa muutos voi jäädä osittaiseksi. Osittainen muutos jättää mahdollisuuden vanhaan paluuseen. Johtajien tulee luoda visio muutoksesta. Ihmisten on helpompi mukautua muutokseen, kun heillä on selkeä kuva tulevasta. Johtajien tulee jakaa visionsa muille oman toimintansa kautta. Toisessa vaiheessa on tarkoitus toteuttaa muutos. Johtajan tulee saada muut henkilöt toimimaan visionsa puolesta. Johtjana kannattaa asettaa lyhyen tähtäimen maaleja jotka helpottavat  ihmisten motivoinnissa. Johtjana tulee vakauttaa jo saavutetut muutokset ja samalla kehittää uusia muutoksia lopullisen vision saavuttamiseksi.  Kolmannessa vaiheessa johtajan tulee tehdä muutoksista pysyviä ja osoittaa että uudet työkalut ja mallit ovat tulleet jäädäkseen. 

Artikkelissa esitellään myös tehokkaan johtajan ominaisuuksia ja tehtäviä ryhmässä. Johtajan tehtäviin lukeutuvat organisointi, innovointi, arviointi ja viimeistely. Kirjoittajat mainitsevat myös ominaisuuksia, joista tehokkaalle johtajalle on hyötyä. Johtajalle hyödyllisiä ominaisuuksia ovat kärsivällisyys, joustavuus, päättäväisyys, taktiikka, kommunikointi taidot, huumorintaju, auktoriteetti ja asiantuntemus. Ohjelmistotuotanto projekteissa ongelmaksi usein muodostuu johtajan johtamistaidon puute. Johtajan tehtäviin muutosten hallinnan näkökulmasta on oleellista koota tehokas ryhmä, jota hän johtaa. Tehokkaan ryhmän ominaisuuksiksi artikkelissa mainitaan yhteinen maali, avoin kommunikaatio, rakentava konfliktien ratkaisu, keskinäinen luottamus, ryhmän henkilöiden välinen tukeminen ja luottamus.


% --- Back matter ---
%
% bibtex is used to generate the bibliography. The babplain style
% will generate numeric references (e.g. [1]) appropriate for theoretical
% computer science. If you need alphanumeric references (e.g [Tur90]), use
%
% \bibliographystyle{babalpha}
%
% instead.

\bibliographystyle{babplain}
\bibliography{references-fi}


\end{document}